\documentclass[12pt]{article}

\usepackage[utf8]{inputenc}
\usepackage[T1]{fontenc}
\usepackage[margin=2cm]{geometry}

\title{Entretien avec Denise Pumain}
\date{15 Décembre 2017}

\begin{document}

\maketitle

\paragraph{JR}

Bonjour Denise Pumain. Merci beaucoup de répondre à notre invitation pour essayer de répondre à des questions sur l'idée de coéovlution entre reseaux de transport et territoires. Alors est ce que je pose un peu le contexte, ou je rentre directement dans le\ldots Oui bon juste pour rappeler un peu les bases - bon j'ai pas besoin de contextualiser en fait, je me disais si c'est retranscrit après, pour les histoires d'effets stucturants, mais les gens peuvent se référer - vous ca vous dérange pas ? \textbf{DP} non pas du tout. \textbf{JR} Donc le sujet general auquel je voudrais arriver après plusieurs question c'est lidee de la coevolution, et en particulier des reseaux et territoires, mais ca pourra etre plus large dans la theorie evolutive, du coup on va commencer par un point particulierm qui est ce fameux debat des effets strcuturants, effet structurant des infrastructures de transports - ca remonte à - bon du coup je suis obliger de faire du contexte - ca remonte a peu pres aux annees 70, où il y avait eu ces papiers d'Alain Bonnafous et tout ca, sur les methodologies de detection des effets structurants, et ca avait fait beaucoup de debat car ca avait ete utilisé par les acteurs publics pour justifier des choix d'infrastructure, y'avait un peu une incomprehension, il ont jamais dit qu'il y avait des effets systematiques, ca a ete mal interprete, mais du coup ya eu beaucoup de literature polemique, Offner et tout ca en 93, et justement en 2014 l'Espace Geographique, un numero special, et donc la on a vu qu'il y avait plein de point de vue differentes, selon les echelles, qu'on regardait et tout ca. Donc ca c'est pour donne rle contexte tres global. Donc juste une question un peu naive, idiote : est-ce que pour vous il y a des effets structurants, est-ce qu'on peut dire scientifiquement, et apres socialement enfin societalement, est ce que le terme ?


\paragraph{DP}

Alors qu'il y ait un debat, je pense, scientifiquement, oui, le debat est encore ouvert, et en cours, et on va peut etre voir socialement, car c'est le plus facile : chaque fois qu'on discute de l'implantation d'une nouvelle infrastructure, y'a toujours debat entre des acteurs, qui - alors peut etre parce qu'ils imaginent que les transports ont un effet structurant sur les activites, mais en general, le plus souvent les acteurs demandent l'infrastructure, avec deux types de raisons : soit il s'agit de consolider un avantage que l'on a deja en terme d'accessibilite, centralité, ou bien, inversement, il s'agit de rattraper une injustice, un ecart, par rapport a ceux qui beneficient de meilleur acces, meilleurs infratsructures, donc il y a toujours dans le processus social une rivalité, competition des acteurs des territoires, de l'economie, qui entendent beneficier des avantages supposés des infrastructures. Donc un processus social de veille et d'attente à l'egard de ces innovations, plus generalement, on constate le meme genre d'attitude et de processus quel que soit le genre d'onnovation. 

\paragraph{JR}

Oui du coup c'est pas que pour les transports- mais quand meme pour les transports on a l'effet specifique de l'accessibilite

\paragraph{DP}

alors pour les transports oui, on attend des retombees de l'accessibilite, qu'il s'agisse de beneficier d'une clientele plus importante, ou de pouvoir beneficier de tout un ensemble d'inputs plus facilement. Du cote scientifique, le debat n'est pas tranche, parce qu'en fait, le debat se situe generalement a plusieurs niveaux, il y a d'une part l'accessibilite brute, qui quelle que soit le type d'infratsructure a tendance a s'ameliorer, a augmenter, pas toujours aussi rapide dans tous les lieux, parce que les infrastructures sont couteuses, donc un reseau n'irrigue pas la totalite d'un territoires, jaqmais, et donc il y a des differentiels d'accessibilite qui se creeent entre les lieux, et ces differentiels d'accessibilite, Anne Bretagnolle l'a bien montré, deviennent de plus en plus importants au cours du temps au fur et a mesure que l'on elabore des moyens de transport qui ont des vitesses de circualtions plus fortes mais aussi des points d'entree plus grands - mais vous savez ca tres bien. alors ce que je trouve interessant, en tout cas dans un debat dans lequel je m'etais engage a propos des chemins de fer, y'avait eu plusieurs episodes, l'analyse que j'avais faite sur l'arrivee du chemin de fer, et la croissance urbaine au 19e siecle, en France, j'avais observé que le chemin de fer s'etait d'une part plutot adapte au reseau urbain preexistant plutot qu'il n'avait cree de nouvelles centralités, ca pratiquement pas, et d'autre part la croissance urbaine etait souvent prealable a la date d'arrivee du chemin de fer. Alors bien entendu a l'epoque je n'avais pas les moyens ni informatique ni technique de verifier cela en tenant compte par exemple du decalage a partir du moment ou la gare est projetee donc engendre deja des effets d'appel sur la population, donc c'etait des conclusions fragmentaire,, hypothetiques. En retravaillant les bases de donnees et sur une periode un peu plus longue, Anne Bretagnolle avait cru trouver, que le chemin de fer avait eu d'avantage d'effets discriminants et structurants que je n'avais imagine - \textbf{JR} c'etait l'idee du renforcement \textbf{DP} avec l'idee du renforcement des centralites par les nouvelles infrastructures, parce qu'historiquement on observe qu'elles sont captees par les villes qui sont deja les plus greandes, et donc meme si par contrecoup les petites villes beneficient d'un gain d'accessibilite, il reste bien inferieur a celui dont les grandes villes beneficient, et plus recemment, avec d'autres methodes, que je n'ai pas suffisamment approfondi pour en dire quelque chose, Arnaud Banos, Thevenin, ont montre qu'en fait il semblait bien que s'agissant du chemin de fer et de la croissance, le chemin de fer n'ait pas eu d'effet determinant sur les differentiels de corissance entre les lieux sur toute cette periode depuis le milieu du 19e siecle. Donc la dedans en fait, ces resultats ne sont pas aussi contradictoires, je pense qu'on a, alors peut etre que vous appelez ca une coevolution, mais ce serait presque une codynamique entre d'une part entre les differentiels de croissance dans les territoires, qui sont dus a toutes sortes de raisons qui ont toutes sortes de composantes, solde naturel, attraction migratoires, investissement industriuels ou transformation des economies, et la construction des infrastructures, elle peut etre est quand meme un peu suiveuse, mais suiveuse pour des raisons d'anticipation de profit. Et ces anticipations de profit en fait conduisent a installer les liens d'infrastructure entre les poles qui vont rapporter le plus en terme de traffic de flux, et donc en privilegiant le profit on renforce les dynamiques territoriales anterieurs d'une certaine facon, tout en les renfiorcant en meme temps, on ajoute des possibilites de connection vers des lieux qui etaient moins nombreux, plus lointains, on augmente la portee, et on produit, on provoque, cette espece de contraction spatiale sous l'effet de la'ugemntation des vitesses.

\paragraph{JR}

D'accord. donc c'est tres interessant du coup, oui d'avoir la chronologie du\ldots Donc ca ca serait le point de vue des geographes, et plus particulierement des geographes theorie eovlutive, est ce qu'il y aurait d'autres\ldots

\paragraph{DP}

oui en tout cas on est en plein dans la theorie evolutive, dans ces effets de feedback entre letat d'un systeme de peuplement et son developpement ulterieur.

\paragraph{JR}

Et du coup, la, en fait j'ai le sentiment qu'il n'y a pas beaucoup d'autres approches geographiques qui ont regarde ce probleme, j'ai l'impression que c'est directement dautres dicispclines : planification, economie, geographie des transports, economie geographique, geographie economique, mais geographie pure, est ce que vous avez en tetes d'autres. Alors en mettant de cote, oui alors c'est biaise, je mets de cote ce qui est purement qualitatif, typiquement des analyses de jeu d'acteurs, parce que la on pourrait reveler des effets.

\paragraph{DP}

Oui alors moi le texte contre lequel je m'inscrivais quand j'ai ecrit en 82 cette article dans les annales de Geographie, cetait des ecrits de Pierre Georges, qui indiquait mais parce que cetait un geographe preocupe d'economie, qui fustugeait un peu la frilosite de certaines villes, de certaines bourgeoisies urbaines qui avaient refuse l'installation du chemin de fer, Tours et Orleans est un des classiques de la geographie de la circulation a cette epoque, et en montrant que le refus de la nouveaute etait prejudiciable au developpement urbain. Donc sur des cas precis, bien entendu, donc comme je vous le disais, moi l'information que je citais c'est Lepetit, qui a traville sur les reseaux a une epoque anterieure aux chemins de fer, qui avait constate, au moment de l'atblissement des routes pavées, royales pavees, qui donc permettaoent d'unifier des itineraires qui vaticinaient jusque la a cuase de l'etat des routes non pavees, qui pouvaient etre endommagees par des inondations, ou connaitres differents obstacles, donc l'etablissement d'une route pavée, allait raccourcir en temps les itineraires, tout en securisant l'infrastructure, et a cette occasion la, en epluchant la litterature, les archives , des lettres qui remontainent vers des intendants, des prefets, il avait constate qu'il y avait deux types de demande pour avoir acces a l'infrastructure, pour que les infarstructures accedent au localites. 

\paragraph{JR}

Alors ca ca pose encore un autre probleme, sur le temps long est ce qua differentes epoques il ya surement eu differentes dynamiques, differentes regimes.

\paragraph{DP}

Alors il ne faut plus regarder la question au cas par cas, comme on vient de le faire, mais sur le temps long, l'effet des facilites d'acces sur les implantations humaines est evident, patent. Que ce soit l'implantationd es routes de la soie qui contournent l'Himalaya par le nord ou par la voie maruitime, que ce soit l'importance du couloir Paris Lyon Mediterranee en France par exemple qui suit les tendances hydrographiques cconnectees, en etant a l'abris. Alors les facilites naturelles de circulation on ete des canaliseurs des implantations humaines en tant que voie de facilitation des echanges, et donc les retonbees du commerce, qui suivait ces implantations plus faciles, ont un effet boule de neige sur le develloppement des implantations, des etablissements humains, et de cette facon on enclanche une dynamique differentielle entre les lieux accessibles et ceux qui le sont moins. \textbf{JR} Et du coup avec la dependance au chemin, on va commencer a selectionner une route particuliere. \textbf{DP} En effet il a une tres forte dependance dans ces dynamiques la, et alors cest ce qui a pu par la suite suciter des polemiques en terme de detremninsme physique sur le developpement en geographie, qui bien sur a eu sa part, mais continue de l'avoir indirectement avec ce que j'appelle cet enchainement historique, cette dependance a la trajectoire des passes. \textbf{JR} Oui parce quand on dit determinisme, on a en tete un vrai determinisime, une stabilite, alors que c'est un determinisme chaotique. \textbf{DP} Oui voila, et de plus ils ont en tete un effet immediat, de la contrainte qui peserait encore, alors qu'evidemment ils ont raison de souligner qu'elle n'est plus ce qu'elle a ete, mais d'une certaine facon on la retrouve. Et donc meme chose pour les reseaux artificiels, pour les reseaux d'infrsatructures de transport implantés, dans les dybamniqeus actuelles, le fait qu'il y ait eu des connections effectuees par certaines infratsructures continue d'avoir des effets sur le temps long, absolument.

\paragraph{JR}

Alors du coup le fameux debat est ce que ca ne serait pas juste un malentendu disciplinaires, on ne regarde pas les memes echelles, on parle pas des memes objets - \textbf{DP} Certainement. oui - \textbf{JR} Est ce que vous pourriez, je sais pas si c'est possible, et si ca a deja ete fait, de faire une cartographie disciplinaire du debat - qui se positionne comment - de votre point de vue.

\paragraph{DP}

Alors de mon point de vue, il y a plusieurs types d'acteurs qui ont interet a defendre l'idee que l'equipement est structurant - \textbf{JR} Ah oui la on revient, pas forcement scientifique donc. \textbf{DP} Pas forcement scientifique, qui eventuellement s'appuie sur des connaissances supposees des acteurs. Donc ceux qui investissent dans la constructionm ceux qui construisent, ceux qui financent la construction, et ceux la contribuent doubleemnt parce que l'investisement s'installe dans des lieux ou un retour rapide et important a court terme est attendu - \textbf{JR} Oui a courte echelle - \textbf{DP} Voila, a court terme tous ces acteurs ont interet a faire comprendre l'importance du\ldots mais aussi tous les eventuellement futures beneficiares de l'installation, soit dans une perspective de rivalite avec d'autres lieux, soit dans une perspective de gains electoraux pour une action benefique a un ensemble d'acteurs economiques locaux, donc les politistes appelent une coalition de croissance, ces ensembles d'acteurs qui sont prets a manifester un interet collectif et qu'ils presentent comme general, pour l'investissement dans l'infrastructure.

\paragraph{JR}

Est ce que vous pensez du coup que ce type d'acteurs peuvent avoir une retroaction sur les disciplines scientifiques, et la recherche qui est faite.

\paragraph{DP}


Alors la recherche qui est faite, bon elle est souvent un peu critique par rapport aux raisons invoquees par les acteurs, mais en meme temps elle doit connaitre ces raisons pour comprendre et interpreter sociologiquement et mene economiquement les observations, donc il est bien possible qu'il y ait, la contamination elle est la aussi reciproque, je pense que les investisseurs sondent les connasisances scinetifiques pour tirer parti de l'etat de l'art, des connaissances passées, et anticiper, meme quand ils essayeint de creer de la nouveaute, et les scientifiques qui connaissent ces demarches, utilisent ces arguments, ces argumentaires, pour leur propre rehcerche, pour montrer la verite des mecanismes du champ - \textbf{JR} Ah oui parce que si on est integre dans le champ, en fait du coup on est pas forcement conscient, on peut se retrouver a integrer, interioriser en quelque sorte. \textbf{DP} Alors c'est souvent inconscient, lorsque, et la c'est pas une critique mechante, quand des specialistes tres exterieurs au champs d'investigation, je pense a des gens qui feraient des, les gens qui ont fait les premiers jeux serieux, ceux de la game theory, qui etaient eventuellement des mathematiciens ou des physiciens, souhaitent appliquer leurs intiotions de dynamiques particulieres a des sujets sociaux, les mnecanismes qu'ils imaginent souvent sont deduites dans un premier temps au moins, de leur intuition, et donc cette intuition ne correspond pas toujours au savoir developpe et acquis par les sciences humaines et sociales, qui ont une experience d'observation et de critique. Donc dans ces cas la il y a  effectivement un decalage entre un etat de la science acquis dans un champ et une mise en oeuvre de techniques sophistiquees, methodologiquement eventuellement interessantes, mais sur des premisses moins assurees. 

\paragraph{JR}

Theoriques oui. Donc la on revient un peu a un probleme epistemologique d'adequation du coup si on veut adapter des nouveaux outils mais qu'on a pas le bagage theorique.

\paragraph{DP}

Oui disons que ca fait perdre du temps a tout le monde, peut etre si ces physiciens matheux avaient passe plus de temps a bien entrer dans le champ, ils auraient produit des modeles plus adpate plus efficaces et plsu integrateurs, ils auraient moins provoquer la defiance de ceux qui pretendent a coup de methodes qualitatives, d'arriver au meme resultat, mais en etant, en ayant pas la meme capacite de convaincre, donc collectivement c'est un peu une perte - \textbf{JR} Oui la fonction d'utilite generale, ils auraient pas publie dans Nature, ya ce cote aussi. \textbf{DP} Oui voila alors en etant plus riche et moins propres, leur chance de publier dans des grandes revues diminuait. 

\paragraph{JR}

En fait c'est amusant la on regarde un sujet precis mais on arrive toujours a des problemes epistemologiques

\paragraph{DP}

Oui et c'est pourquoi on estm au bout d'yun certain temps en recherche, on est condamme, on arrive a faire un peu d'expistemologie, parce qu'on rencontre des problemes qui ont ete souleves depuis longtemps - \textbf{JR} Oui sinon on tourne en rond, et il faut sortir par le haut du - \textbf{JR} Oui et ca permet aussi de mieux comprendre pourquoi il peut y avoir des point de vue aussi differents et contradictoires, en meme temps quamd les georgaphes se confrontent a l'economie pure et dure, mainstream, il y a des incomprehensions importantes qui se produisent parce que effectivenent on ne se pose pas les memes questions, meme si le vocabulaire employe - \textbf{JR} est plus ou moins le meme - \textbf{DP} c'est le cas. \textbf{JR} C'est pas du tout les memes ontologies, on parle pas des memes objets. \textbf{DP} Absolument, et les problemes qu'on pose n'ont rien a voir.


\paragraph{JR}

D'accord. Alors on va revenir au sujet concret. On a commence tres specifique, sur l'histoire des effets strtucturants. Est-ce que - alors la c'est une question qui peut ne pas avoir de reponse, moi j'y repondrais plutot negativement, mais du coup j'aimerais bien avoir votre point de vue la dessus, est-ce que on aurait une classe de probleemes un peu plus generale que cet histoire d;effets et de feedbacks dqnas un sens et dans l'autre, donc qui seraient les interactions entre reseaux de transport et territoires, certaines classes de porcessus de probemes, qui ne releveriante pas du coup du premier debat. du coup du point de vue de la theorie evolutive. \textbf{DP} Du point de vue de la theorie evolutive, sur le feedback reseaux territoires ? \textbf{JR} Oui du coup j'ai oublié de preciser aussi - qui ne releveriant pas de la coevolution, qui seraient entre les deux. parce que la j'essaye d'arriver a la coevolution a la fin, une vue plus integree, dans l'histoire des dynamiques structurelles du systeme, qui ne seraient pas coevoltuion, mais qui serait un peu plus riches qu'effets structurants - estce quil y aurait une etape intermediaire ?

\paragraph{DP}

Oui alors. Plus general que l'effet structuirant. L'effet structurant c'est l'impact que les reseaux ont sur le temps long, et donc l'accessibilite qu'ils propagent, ont sur le territoire. Et la on a quand meme montre que ca avait d'une part des effets de localisation, et d'autre part des effets de creusement progressif des inegalites, dans la mesure ou les vitesses augmentent, et les retonbees de leurs effets aussi, et dans la mesure ou il y a une dependance a la trahectoiure anterieurem il y a a long terme des effets structurants en termes de localisation des reseaux et ne termes du developpement. Alors sur kle plan plus general des relations entre reseaux et territoires, du point de vue statique, il y a une interpenetration, et tout territoire est le dual d'un reseau et inversement - \textbf{JR} Alors oui jh'avais un peu en tete le Offner Pumain, et un peu en lien avec Gabriel Dupuy - \textbf{DP} Absolument,m et aussi ce que vous pouvez trouver chez les biologistes, avec tous les reseaux de transport d'energie, ou de sang de vascularisation, et le tissu qui les enveloppe,m donc il y a une dependance fonctionelle entre le territoire et la criculataion, et inversement. \textbf{JR} Donc les ontologies sont deja couplees a la base en fait. \textbf{DP} Les ontologies fonctionelles. Alors y'a un geographe qui avait fait ca en statique un petit peu, mais cetait assez facile a expliquer pour le grand public ou les etudiants, il s'appelait Phil Brick. Vous avez vu ce papier qui date de 1957 ? \textbf{JR} Ah non. \textbf{DP} Alors cetait un papier qui lui se situait dans l'organisationmulti scalaire des territoires, et dans le vieux debt, qui a existe aussi, ca parait etrange aujourd'hui, entre regions homogenes et regions polarisees. en geographie -  \textbf{JR} Homogene, du coup un peu a la Christaller ? \textbf{DP} Non homogene correspondait a une unite paysagere, a une unite agricole, une petite region agricole on vas dire, voila ce qu'est une region homogene, 
% 28:11


















\end{document}
