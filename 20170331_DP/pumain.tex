\documentclass[12pt]{article}

\usepackage[utf8]{inputenc}
\usepackage[T1]{fontenc}
\usepackage[margin=2cm]{geometry}

\title{Entretien avec Denise Pumain}
\date{31th March 2017}

\begin{document}

\maketitle

\paragraph{JR}

Bonjour Denise Pumain (\textbf{DP : } Bonjour), merci donc d'avoir bien voulu répondre à nos questions sur la théorie évolutive des villes. Juste pour fixer un peu le contexte, on propose un cadre de connaissances pour revisiter la démarche de la Géographie Théorique et Quantitative, et on pense que la théorie évolutive des villes, que vous avez développé maintenant depuis plus d'une vingtaine d'années, serait une illustration parfaite de ce cadre de connaissances. Je vous demanderai de développer certains aspects de l'histoire, et peut être de l'avenir de cette théorie évolutive. Pour commencer, une question très générale, mais pour situer les choses : qu'est ce que pour vous, vous apeleriez théorie évolutive, et est-ce que vous en avez une vision synthétique et unifiée aujourd'hui ?


\paragraph{DP}

Vous donnez d'emblée, à froid, une définition, hors contexte ! C'est une théorie géographique, en ce sens qu'elle s'intéresse à l'organisation des villes dans des territoires, à différentes échelles, et la principale ambition de cette théorie c'est de rassembler l'essentiel des faits connus sur ces objets en les plaçant non pas dans dans une théorie disciplinaire dans laquelle il y aurait des présupposés d'équilibre, de structures statiques arrêtées, mais dans une perspective spatio temporelle, qui permet de suivre ces objets sur de très longues périodes de temps, depuis leur émergence il y a 5000 ans, avec une attention d'une part aux principaux facteurs structurants de ces organisations, qu'on décrit à l'aide des termes de systèmes, et puis une attention également à signaler, observer les principales bifurcations qui marquent l'histoire de l'évolution de ces systèmes. Voilà, et c'est dans ce sens là qu'on imagine que les concepts des systèmes complexes peuvent nous aider à retrouver quelques propriétés des systèmes urbains. Essentiellement, considérer, du point de vue géographique les villes comme un systèmes dans les systèmes urbains.

\paragraph{JR} Donc là c'est Berry \ldots

\paragraph{DP}

C'est Berry mais considérablement enrichi et revisité, parce que pour nous , la ville est un objet complexe mais qui n'est pas un système entièrement autonome, chaque ville est insérée dans des réseaux, qui parfois sont très largement déterminants pour son futur devenir. D'où l'intérêt d'observer des grands ensembles de villes interconnectées, et qui co-évoluent, de façon à mieux comprendre les particularités de chaque ville.

\paragraph{JR}

Du coup, juste pour rebondir, et ça sera peut être une petite digression : donc on voit l'importance du cadre de la complexité et des Sciences des Systèmes Complexes dans cette théorie, du coup il y a bien l'idée de transcender les disciplines, car au début vous avez opposé à un point de vue disciplinaire, l'équilibre et tout ça, ça devait être en pensant à l'économie (\textbf{DP : } Tout à fait), l'économie mainstream qui ramène toujours tout à l'équilibre (\textbf{DP : }Exactement). Du coup est-ce qu'il y a aussi un géographie de l'équilibre, ou c'est complètement dépassé, c'était à l'époque de\ldots

\paragraph{DP}

Il y a eu quelques tentatives, notamment avec les cycles d'érosion, et l'idée géomorphologique, que on aboutissait nécessairement à une plaine, puis ce qu'on a appris de la tectonique a montré qu'on était là aussi dans des systèmes toujours en déséquilibre, contrairement à cet espèce d'arrêt stationnaire à un moment donné, qui n'est pas l'aboutissement nécessaire, je pense que socialement plus personne ne prétend qu'il y a équilibre, seulement c'est un instrument de pensée à partir duquel les économistes construisent leurs catégories, leurs raisonnements, hélas parfois leur modèles, pour les appliquer, alors que la recherche de cette optimisation ne peut être pour moi éventuellement qu'une construction de scenario, acceptables ou indésirables, qui peut tout au plus aider à penser certaines préconisations sur des durées très courtes, et en tout cas ça n'est jamais une construction suffisamment intéressante pour expliquer l'observation des évolutions.


\paragraph{JR}

Alors, mais en fait les, enfin c'est une impression que j'en ai - c'est encore une digression, je ne fais pas très bien les entretiens, mais ça me fait penser un peu à ça, et c'est important je pense - la plupart des tous ces travaux d'économistes et en fait aussi je pensais au transports, car j'ai fait un travail sur l'équilibre Static User Equilibrium en transports, ce gens là privilégient ces approches parce qu'ils veulent des systèmes qu'on peut résoudre analytiquement, qu'on peut maitriser, on peut trouver des solutions, et j'ai eu encore un débat avec mon ami qui est économiste l'autre jour, ils ont un modèle tout simple, et juste ils n'arrivent pas à le résoudre analytiquement, mais l'ordinateur trouve des solutions numériquement, mais ils ne veulent pas les accepter, ils veulent la résolution analytique, et ils sont bloqués depuis deux mois sur la résolution analytique, c'est assez incroyable, mais (\textbf{DP : } Certains mathématiciens, autrefois, en effet\ldots) quel est - alors c'est bien parce que ça nous fait rebondir sur l'histoire de co-évolution des outils et des méthodes avec la théorie, est-ce que la géographie a réussi à dépasser cette dépendance, euh, à un espèce d'idéal, je sais pas déjà d'où vient ce besoin d'avoir une résolution analytique, mais est-ce que la géographie a réussi à le dépasser, et si oui pour quelle raison, est-ce qu'il y a une particularité géographique qui fait qu'on est pas dépendant de ce besoin.

% 7min35


\paragraph{DP}

Je pense que la théorie économique mainstream, enfin ça remonte à Walras a un certain mécanicisme, en même temps que une volonté de pouvoir formaliser par les mathématiques la partie la plus facile à numériser du contenu social, qui est sa dimension comptable, laquelle était déjà établie empiriquement depuis très longtemps, par les comptabilités à double entrée des entreprises qui ont été mises au point plusieurs siècles avant, donc y'a tout ce passé, plus les intérêts que vous signalez actuellement pour disposer de résultats tangibles sous forme de calcul, et d'évaluations de coûts, je ne suis pas en train de dire que l'économie n'a pas sa raison d'être quelque part pour un grand nombre d'opérations, dont certaines éventuellement touchent les villes. Alors nous cette approche je ne pense pas qu'on puisse dire que nous l'ayons dépassée, dans la mesure où elle ne nous a jamais rendu de très grands services, parce que pendant très longtemps la géographie a été plutôt littéraire et descriptive, et quand elle est passée au stade de l'analyse quantitative, elle avait déjà défini des objectifs de recherche qui admettaient les inégalités, la diversité comme un objet non pas à récuser non pas à réfuter non pas à reléguer dans un petit résidu de modèle, mais au contraire comme une substance qui faisait l'intérêt. Même un géomorphologue très quantitativiste, Pierre Birot parlait de la beauté de ce que jamais l'on ne verra deux fois. Donc l'unicité des objets géographiques, qui a fait le plaisir des grandes découvertes, des récits de voyage, qui aujourd'hui fait le bonheur de beaucoup de touristes, c'est quelque chose qui demeure le-un des stimulants ludiques de la géographie dans la recherche, et aussi un des ressorts de sa recherche d'explications et de formulation de théories. En ce sens là, on peut comprendre les réactions négatives de certains géographes à l'encontre des premiers essais de formalisation mathématique en géographie, le plus drôle c'est Walter Christaller inventeur de la théorie des lieux centraux, se moquant de Zipf ou Auerbach, en disant que la loi rang-taille n'était qu'un simple jeu de nombres, c'est amusant car Christaller s'est fait critiqué de son côté pour être, pour donner un schéma très caricatural de l'organisation des - enfin jugé comme caricatural par certains - de l'organisation des villes, et Pierre Georges encore dans les années 1950 ou 60, récusait l'idée qu'on ai besoin de faire une géographie des supermarchés. Donc la dimension comptable et analytique en géographie, elle a pu être contestée, et elle a été récusée, souvent vue comme un danger, précisément de contraindre ou d'enfermer les territoires dans des fonctionnements qui auraient été trop normatifs, trop homogènes, mais la géographie ne s'est jamais donné pour objectif de proposer quelque chose d'aussi fort que le modèle du marché offre et demande, enfin d'identifier la variable-les variables clés qui devraient rendre compte d'un ensemble de faits suffisants.

\paragraph{JR}

Donc jamais de loi universelle, mais identifier des processus qui reviennent plus ou moins régulièrement, ...

\paragraph{DP}

On parlait de régularité plutôt que de loi, c'est une discussion que nous avons encore avec Michael Storper qui refuse absolument l'idée qu'il y ait des lois, je pense qu'il oublie que le mot loi c'est un emprunt des sciences dures au monde social, puisque les lois sont énoncées par des puissances, mais blague à part, je pense que les, ces régularités elle sont, comme toujours en sciences humaines et sociales, elles sont nécessairement contextualisées, et celles qu'on est capable d'énoncer à un niveau si élevé d'abstraction qu'elles sont vraiment universelles, elles sont très rares, et quand on les énonce, c'est sous la forme verbo-conceptuelle, à la Tobler, la première loi de la géographie qu'est ce que c'est : tout interagit avec tout, mais deux choses proches ont plus de chances d'interagir que deux choses éloignées, tel quel l'énoncé n'est pas, n'est qu'en partie falsifiable, c'est déjà un énoncé falsifiable, donc c'est vraiment une loi, une théorie au sens très fort, pour la géographie, bon et ensuite on se contente de lois bien mieux contextualisées, je crois.

\paragraph{JR}

Parfait. Alors je propose qu'on revienne plus en détails sur l'histoire de la théorie évolutive, et son contexte. Alors je propose une grille de lecture peut être un peu simple, mais temporelle en fait, pour clarifier un peu certaines, certains points qui auraient pu être particuliers, ou certains moments clés. Alors voilà, là j'ai essayé de faire, c'est pas fini donc, mais une, je vous montre l'image, un peu une synthèse, c'est pas complet du tout parce que pour l'instant j'ai pris les références que j'avais dans ma biblio à moi, et je sais que j'ai plein de papiers à ajouter là dedans, mais donc dans le temps, et dans les domaines de connaissance, essayé d'insérer les publications, et soit les notions, soit les modèles, soit les bases de données, donc en rouge c'est plus des notions, bon c'est pas unifié, je referai, et aussi des personnes, alors j'ai pas du tout l'aspect social pour l'instant, mais les noms des personnes dans les publications, mais c'est vrai qu'on pourrait mettre d'autres entités qui sont les personnes en fait, où elles ont évolué tout ça, euh voilà...

\paragraph{DP}

Vous voulez que je commence par le commencement ?


\paragraph{JR}

Oui du coup, alors ça c'était juste pour peut être un peu situer les choses, et je suis pas sûr aussi d'avoir commencé au bon, est-ce que c'est l'article séminal ou pas, donc voilà mon premier point c'était les origines, quelles auraient été les précurseurs, vous avez déjà donné un peu la réponse au début en parlant des théories des systèmes complexes tout ça, des précurseurs, est-ce qu'il y a eu un article ou des actes fondateurs, un moment clé où on s'est dit maintenant on a une théorie évolutive, est-ce que le fameux (Pumain, 97) c'est le...

\paragraph{DP}

Le Pumain 97 c'est l'article séminal (\textbf{JR} d'accord) qui enregistre l'expression de la théorie évolutive, mais je pense qu'on peut, enfin, remonter beaucoup plus loin, sur mon parcours personnel, mais je vais vous le faire d'une manière qui correspond bien à votre démarche intégrée, parce que je trouve ça amusant, je l'ai jamais fait encore, mais bien entendu les précurseurs ils sont très nombreux, dans l'histoire des idées et des travaux sur les villes, donc là un jour je vous referai tout l'état de la question, mais ça pourrait être très long, trop long sûrement. Mais ce qui est amusant, c'est que c'est un peu par accident que j'ai commencé à travailler sur les villes, puisque avec Marie-Claire Robic, et à l'instigation de Philippe Pinchemel, avec qui nous souhaitions faire notre mémoire de maitrise en 1968, ça a commencé par une question de données, une géographe indienne de passage avait découvert à l'INSEE des tableaux d'information tirés des recensements qui avaient été construits en réponse à une question posée aux résidents : où habitiez vous au recensement précédent ? Ca couvrait la période 1954-1962, et ça donnait une description assez précise de l'âge et des lieux d'où venaient les populations migrantes, à l'intérieur de la France, et c'était distribué assez précisément en terme de localisation géographique, et intelligemment parce que ça avait été construit autour des agglomérations urbaines, déjà en dépassant l'échelle communale, et pour une agglomération donnée de plus de 50000 habitants, on pouvait identifier l'origine des migrants, des communes rurales du même département, de tel ou tel autre département, etc., bon on a réaggrégé par la suite, et donc on a pu enregistrer les effectifs de toutes les personnes qui s'étaient déplacées entre deux recensements, et avec pour point d'arrivée ou d'origine nos agglomérations de plus de 50000 habitants. Et là dessus donc, l'intérêt pour les villes a émergé d'une disponibilité de données pour en parler, pour travailler dessus vec quelque chose qui d'emblée allait, parlait de changement, puisque c'était une population qui se déplaçait, d'emblée on a été confrontés aux outils et aux méthodes, les outils une machine à calculer pour faire les multiplications et les divisions avec un bruit de moulin à café, je me demande pas si on le retrouve maintenant dans les cafés branchés, un bruit rrr de moteur. Une toute petite, comment on va dire, sensibilisation statistique qui nous permettait de deviner que moyenne et écart-type avaient de l'intérêt pour étudier les distributions, et une énorme frustration méthodologique, parce que moi j'avais des profils de migration, en fonction de l'origine plus ou moins lointaine, et Marie-Claire qui travaillait sur les structures démographiques avait des profils d'âge, comment on fait pour analyser un profil dans son ensemble, et pas seulement des variables deux à deux, avec un coefficient de correlation. Et ça à l'époque, on était quand même, rétrospectivement j'étais fâchée que personne ne nous dise : voilà les multivariés ça existe, les ordinateurs ça commence à arriver, on en a besoin... Donc cette frustration ; ensuite on a prolongé le travail, donc réussi avec un mémoire de maitrise rédigé en commun, une innovation pour l'époque, puis passé le concours de l'aggregation, donc une interruption, mais quand on est revenues ensuite comme assistantes aggrégées, dans une université, on avait, moi j'avais appris le fortran entre temps, donc déjà on avait accès à un centre de calcul, et on a eu à répondre à un appel d'offres sur la croissance des villes, qu'est ce qui explique que telle ou telle ville croisse et d'autres pas. Et, tellement le sujet, enfin la science sociale était, on va dire, éventuellement politisée à l'époque, en tout cas sensible à ces démarches, la question c'était est-ce que la couleur du maire, la couleur politique du maire, a une influence sur ces processus. Avec l'idée qu'il pourrait y avoir des personnes plus ou moins favorables à ceci. Et en se penchant sur, en dialoguant avec les personnes en question qui faisaient des statistiques aussi, il y avait une contradiction flagrante entre ce qu'on nous disait ``oui nous avons fait tel programme de construction de logements, oui nous avons été favorables à l'implantation de telle ou telle entreprise, et puis ce qu'on lisait dans les statistiques, qui était qu'on voyait la même chose partout. D'où l'intérêt déjà pour les régularités - pour observer des régularités, ne pas se contenter d'une explication locale, et donc assez vite, je sais plus sous quel déclencheur, en tout cas on est passé à la loi rang-taille et au modèle de Gibrat, sur lequel j'ai travaillé, la dynamique des villes, c'est 1982. Et entre temps, j'avais publié avec Thérèse Saint-Julien, en 1978, les dimensions du changement urbain, et là quand même la plus grosse découverte c'était la co-évolution socio-economique des villes. C'était (\textbf{JR} Donc c'était déjà dans le 1978...) - ah oui oui - (\textbf{JR} Donc la théorie évolutive était déjà implicitement) - y'avait un germe. Déjà une théorie c'est jamais une affaire solitaire (\textbf{JR} oui bien sûr), c'est jamais l'invention d'une personne, et puis c'est quelque chose qui mûrit, forcément un peu lentement, mais de manière complètement intriquée entre données, modèles, observation, publications\ldots Enfin voyez, et vie de laboratoire, échanges extérieurs, puisque chemin faisant, on est allées voir des gens qui travaillaient avec la théorie de l'information, je suis allée voir, je suis allée écouter un jour un exposé sur l'entropie par Peter Allen, plus tard, et je participais aux réunions des géographes quantitativistes qui se formaient aux disciplines des statistiques, des probabilités, enfin toutes choses qui tombaient sous le sens et qui devenaient accessibles à l'époque. Donc c'était un propos liminaire, mais pour vous montrer à quel point cotre idée, quand j'ai lu géographie intégrée, je me suis dit mais pourquoi, bon sang mais bien sûr - non c'est, vraiment je suis très contente que on en vienne aujourd'hui à identifier une démarche qui en fait nous a inspirée, sans que nous l'ayons formulée en tant que telle dès le départ. (\textbf{JR} Une idée mûrit\ldots ) Oui et en même temps, ça nous paraissait nécessaire, alors moi j'ai fait une formation mathelem au baccalauréat, donc les professeurs géographes qui proposaient des énoncés testables, vérifiables, ou partiellement quantifiés, ils m'attiraient ; voir que l'on pouvait - je me souviens d'un cours de Michel Rochefort, qui par ailleurs n'était pas très nourri, mais qui néanmoins nous avait dit, en s'appuyant sur d'autres travaux, ça ça devait être je crois en 67, en cours de licence ; il nous disait bah voilà, dans une ville, on peut avoir jusqu'à 75\% d'emplois industriels mais pas plus, dans les années 50, on est toujours dans cette période % 23min15
où y'avait encore des villes minières, des villes sidérurgiques, et là je me dis ah bon, on peut donc mesurer quelque chose qui parle de l'activité des villes, et donc très vite avec Thérèse on a cherché des données qu'on a trouvé, dans les recensements, pour exploiter les profils économiques, sur lesquels on a écrit donc ce premier ouvrage, 78, qui à la fois faisait un peu la nique au, puisqu'on était chacune dans deux laboratoires différents avec deux directeurs de thèse de doctorat différents, et on osait publier un livre avant d'avoir soutenu notre thèse, c'était un scandale, bref... la dimension sociologique du défi, du challenge, n'est pas complètement absente.

\paragraph{JR}

Oui du coup je rebondis sur le côté du labo, j'avais noté cette question, quel était le rôle des structures de recherche, et particulièrement de Géocités, enfin Géocités est un peu né en relation avec toute cette aventure ? (\textbf{DP} alors...), où il n'y a aucun rapport ?

\paragraph{DP}

Elle s'est construite, si y'a un rapport, en ce sens où elle s'est construite autour de trois personnes, Thérèse Saint-Julien et moi, en même temps Marie-Claire Robic qui était dans notre groupe de travail a été nommée à Créteil, et puis avait choisi de faire de l'histoire des sciences, elle s'était un petit peu éloignée de ces travaux, mais à trois on avait, nous faisions des réunions de bibliographie, de réunions de..., et j'avais entamé un enseignement de méthodes quantitatives dès 72, auquel Thérèse a du se raccrocher vers 75, et je pense que c'est en 76 qu'on est venues ici partager un bureau, avec un professeur d'informatique qui avait été nommé à Paris 1, pour accompagner le mouvement d'enseignement des statistiques dans, pour les sciences humaines et sociales. Et voilà, donc en fait, construire un laboratoire c'était aussi bien une question d'organisation matérielle, puisque les assistants et maîtres assistants à Paris 1 avaient été gratifiés d'un petit bureau de 8 ou 9m$^2$, au quatrième étage, qui devait servir à 40 personnes, et le travail collectif y a démarré parce qu'on s'y retrouvait, Thérèse et moi, tous les matins à 9h, aussi bien pour préparer par avance les cours, ou faire les travaux d'analyse quantitative, et la rédaction d'articles et d'ouvrages, et donc on avait compris, on a fait installer une armoire, on avait compris l'intérêt d'une organisation matérielle, et collective du travail, donc c'est ça qui nous a motivées pour créer ce qui s'appelait d'abord une jeune équipe, et là c'est vraiment le CNRS qui a lancé des programmes permettant l'accrochage institutionnel d'initiatives intellectuelles, et donc ce laboratoire est devenu, on appelait ça d'abord unité associée au CNRS, je crois que ça s'est produit sous l'appellation équipe Paris, c'était en 1984 peut-être, et puis en 92, on, le CNRS nous a demandé de nous regrouper avec EHGO, et c'est là où l'appellation Géographie-cités a été choisie (\textbf{JR} d'accord, EHGO existait déjà par ailleurs\ldots ) ; EHGO existait déjà par ailleurs, avait été fondé je crois dès 76 par Philippe Pinchemel et M. Maulat, qui était un historien, comme un laboratoire de géographie historique et d'histoire de la géographie, qui avait des locaux rue Maller, des locaux de l'université Paris 1, qui était également associé au CNRS, peut être avant l'équipe Paris, et pour des raisons de réduction du nombre d'équipes en SHS, c'est un sujet qui revient régulièrement au CNRS, et puis parce que nos méthodes de travail n'étaient pas si éloignées finalement, on nous a demandé de nous regrouper, et EHGO est venu nous rejoindre rue du Four. Voilà. Entre temps c'est Thérèse Saint-Julien qui avait dirigé l'équipe Paris quand elle a été créée.

\paragraph{JR}

D'accord. Et alors du coup, c'est un... le livre de Lena Sanders sur Synergétique, c'est en 1992 il me semble (\textbf{DP} Oui). Alors quel est le lien avec la théorie évolutive donc ?


\paragraph{DP}

Alors dans les recherches, dans cette découverte de coévolution des villes, qu'on appelait pas encore coévolution, mais de parallélisme, et de transformation des villes en termes d'image de marque et de modernité technique, qu'on avait acté 78, on a cherché des formalisations d'abord mathématiques pour essayer de rendre compte de ce changement, qu'on observait avec des analyses permettant de, des analyses multivariées, permettant de simuler des traj - enfin pas de simuler, mais de construire des trajectoires à partir de quatre photographies transversales, c'est à dire à des moments différents (\textbf{JR} une typologie des trajectoires\ldots ) - voilà. Et ça c'était quelque chose d'un peu insatisfaisant, puisque on avait pas le processus qui conduisait à la transformation. Donc à la recherche de ça, et après avoir entendu Peter Allen dans un exposé à Créteil sur l'entropie, mais qui parlait de modalités de transformation de ces systèmes auto-organisés, en terme d'ordre par fluctuations, beaucoup de petites transformations aléatoires par rapport à la structure qui ne la transformaient pas, mais certaines agissant de façon répétée contribuaient, et ça nous on avait eu l'intuition de cette forme de changement, en observant l'émergence d'un deuxième facteur dans les analyses factorielles, et en comparant l'évolution des trajectoires, enfin ça se faisait à la main et de manière très ad hoc vous me direz, donc Peter Allen proposait des modèles pour analyser ces transformations, mais qui allaient, qui étaient des modèles plutôt intra-urbains ou intra-régionaux. Donc on a accepté de faire le détour, notre premier détour, pour tester la manipulation de ces systèmes d'équations non-linéaires, permettant éventuellement de rendre compte des changements observés. Ca nous a obligé à changer d'échelle, à quitter le système de villes pour passer dans les villes, puisque, et encore on était pas très satisfait, puisque du point de vue statistique, on pouvait seulement pour une agglomération urbaine, avoir l'échelle communale, avec des données socio-économique suffisantes pour renseigner le modèle, donc sur l'agglomération de Rouen on était à quelque chose comme 17 communes, c'était, mais c'était un bon point de départ, sur lequel Lena Sanders a fait sa thèse, et sur lequel on s'est ensemble bien cassé les dents, alors Lena Sanders je l'ai rencontrée en 1982, à San Miniato. San Miniato, c'était une première énorme rencontre de gens qui aujourd'hui, dans tous les domaines de l'observation des villes et des territoires, et en modélisation, sont encore là, y'avait Peter Mallcamp, y'avait Giovanni Rabino qui vient de disparaître, y'avait Regapi, y'avait Ora Regiani, y'avait, Gunther Haag était là, y'avait des américains Dan Griffiths, Leslie Corie était là, on peut avoir des listes complètes, mais c'était vraiment un noeud de rencontre intéressant, et c'est là où j'ai compris que l'Europe était en avance par rapport aux américains, qui étaient tous complètement décontenancés par la présentation de Gunther Haag et Wolfgang Weidlich sur les systèmes auto-organisés, alors que moi j'avais été déjà bien en contact avec donc l'équipe de Prigogine, via la rencontre avec Peter Allen, Michel Sanglier, Prigogine lui-même, via des colloques de dynamique des systèmes à Boston et à Bruxelles. (\textbf{JR} Oui les américains étaient plus dans les stats spatiales\ldots ) - et plus dans le modèle économétrique finalement (\textbf{JR} oui oui oui). Hors Lena Sanders a d'emblée, était aussi intéressée par ça, et elle a bien voulu faire une thèse sur l'application de ce modèle de Peter Allen, donc au début on a ramé beaucoup car on nous avait donné une mauvaise version Fortran du modèle, qui ne pouvait pas marcher mathématiquement, et à force de travailler dessus jusqu'à minuit au centre de calcul au Panthéon, et de retourner à Bruxelles en montrant qu'on avait fait toutes ces expériences qui marchaient pas - ah oui, parce qu'on nous disait au téléphone il faut chipoter, (rires) Michel Sanglier dixit, avec les paramètres etcaetera, bref et découvrant le modèle, ah oui on vous avait donné, mais effectivement cette équation-là\ldots , bon bref, on est revenues avec une nouvelle version - \textbf{JR} c'est amusant parce que c'est la même chose aujourd'hui \textbf{DP} c'est vrai ? \textbf{JR} bah faut touiller les paramètres un peu, ça marche pas ça marche pas, ah oui au fait tout était faux depuis le début, je pense qu'on a les mêmes problématiques qui ressortent - \textbf{DP} Et c'est quelles, des erreurs qui vous avaient été sciaemment données, en connaissance de cause, ou ? - \textbf{JR} bah des fois y'a des biais dans les algorithmes et on le sait pas, puis trois jours après y'a une correction du bug dane le truc, après c'est nous qui avons mal codé une équation\ldots - \textbf{DP} En tout cas ça peut arriver aussi oui. Mais là moi j'étais assez naïve en matière d'équations non-linéaires, j'avais refait un peu de formation mathématique quand même, mais en particulier pour comprendre tout ce qui se disait sur la déduction de la loi rang-taille en terme de maximisation d'entropie, qui est complètement opaque et mensonger à la limite, mensonge par omission, bref. Donc, mais comprendre, enfin anticiper le fonctionnement dynamique d'une équation juste en la regardant, d'autant que comme je l'ai expliqué à Fabrizzio justement hier, qui me disait, mais on rejoint, la formalisation fait se rejoindre la réalité et la modélisation, il me disait mais quelle formalisation, car en fait y'a aucune ambiguïté effectivement si on s'en tient à un formalisme mathématique, ou même algorithmique, en revanche, dès l'instant qu'il est renseigné par des concepts, et des contenus mis sous des paramètres, si ça ne se comporte pas comme, conformément à la signification qu'on y a mise, on a un problème, donc y'a pas que l'implémentation mathématique et informatique de la formalisation, y'a aussi toute une formalisation conceptuelle qui est présente dans le modèle mathématique, manipulé par des - \textbf{JR} une sorte de validation externe en fait ? enfin si on a un paramètre et qu'on lui donne une plage de valeurs où il a le droit d'aller, et - \textbf{DP} oui, et qu'il y obéit pas, ça fait un souci pour le modélisateur de sciences humaines, nécessairement. Et je n'ai compris ça, enfin j'ai identifié le problème très clairement que avec une stagiaire belge qui est venue ici, des années plus tard, et avec laquelle on, qui programmait et qui avait programmé en termes d'équations non-linéaires, avec le modèle de Volterra-Lotka, et on avait décidé de l'appliquer aux échanges centre-périphérie, on a travaillé avec Lena là-dessus, entre une agglomération et sa périphérie, le coeur de l'agglomération absorbant d'abord les communes rurales en périphérie, puis redéversant sa population ; on se disait c'est facile à programmer avec, on connait les modalités qui sont pareille partout ou presque, ou on les fait fonction de la structure d'âge de la population, on a des mouvements migratoires, ça doit être facile. Et en fait on s'est cassé les dents, on arrivait pas à faire en sorte que le x-y, enfin le paramètre d'interaction du Volterra-Lotka, ne contienne que des migrations, que le natalité-mortalité ne contienne que la variation naturelle de la population, c'était in-fai-sable. Donc\ldots - \textbf{JR} En fait il doit y avoir plusieurs solutions équivalentes\ldots \textbf{DP} Probablement, mais en fait il aurait fallu une décomposition multi-agents en fait (\textbf{JR} oui c'est ça en fait) - un décomposition par sous-ensembles, et contraindre chaque sous-ensemble à fonctionner - \textbf{JR} Car sur le modèle agrégé on pourra le calibrer de plein de façons différentes en fait, et on saura pas laquelle est la vraie, enfin celle où l'interprétation\ldots - \textbf{DP} Voilà, on peut pas maîtriser le, ça c'était une belle leçon quand même par rapport aux modèles mathématiques, qui évidemment sont plus propres, plus beaux, plus économes, plus faciles à transmettre, tout ce qu'on veut comme avantages certains\ldots - \textbf{JR} Y'a une publication sur le Lotka-Volterra ? Car je trouve ça assez intéressant - \textbf{DP} Y'a un rapport, donc il doit être ici, j'espère l'avoir gardé quelque part dans la bibliothèque ; j'arrive plus à retrouver son nom. \textbf{JR} Parce que c'est un sentiment que j'ai, aujourd'hui les, quand je parle un peu avec tout le monde et tout, tout le monde a un peu perdu le côté systèmes dynamiques, alors que pourtant il faut toujours l'avoir à l'esprit, que dans certains cas ça peut marcher, et du coup je sais pas, faudrait regarder si, du coup vous l'aviez fait à une époque mais ça a été oublié\ldots \textbf{DP} Alors j'ai, ah oui on a pas recommencé\ldots \textbf{JR} Parce que quand je discute avec les gens qui font du multi-agent, ils ont pas en tête que dans certains cas, en effet un système dynamique ça serait peut être mieux, ou que dans ces cas là simples on peut agréger pour faire un système dynamique, ou là c'est pas un multi-agent mais c'est un système dynamique, on peut généraliser\ldots \textbf{DP} Pour les proies-prédateurs, y'a pas mal de choses de faites quand même\ldots \textbf{JP} Par des Géographes ? \textbf{DP} Des géographes qui travaillent sur, oui dans les Causses, sur les moutons et loups\ldots \textbf{JR} Ah oui oui, de la biogéographie, mais du coup pas appliqué aux villes\ldots \textbf{DP} Ah non, aux villes\ldots \textbf{JR} Aux villes, dynamique des villes\ldots \textbf{DP} Alors aux villes, les systèmes avec équations à la Forrester, y'a eu une application sur Carpentras, faite par un élève de\ldots, un élève d'un économiste, dont le nom ne me reviendra pas, mais bon ça peut se retrouver, mais qui a, et puis par une fille qui s'appelait Christine Alexandre qui avait essayé de faire ça à Toulouse, bref y'a trois petites tentatives comme ça, mais qui, alors on travaillait avec Dynamo, et écrire en Dynamo des interactions spatiales entre un centre et une ou des périphéries, apparemment c'était carrément dur - \textbf{JR} Donc là on revient au outils en fait, au problème de l'outil, il y aurait fallu\ldots Mais peut être que revisiter des choses qui ont été faites dans le passé, avec les nouveaux outils, avec les nouvelles méthodes, très intéressant. D'ailleurs du coup un aparté - en fait j'avais, en fait on pourrait faire ça de manière complètement systématique, en notant toutes les personnes, toutes les références, toutes les notions, et le coder en graphe, et faire un graphe multi-layer (\textbf{DP} oui), et y'aurait peut être des choses qui sortiraient de l'analyse, une analyse un peu quanti de, enfin du coup pas que quanti mais - \textbf{DP} De toute cette bibliographie\ldots \textbf{JR} Voilà en fait, et quantifier la co-évolution, des fois on dit ah bah dans cet exemple là, l'outil a plus servi, du coup % 40min 34


































\end{document}
