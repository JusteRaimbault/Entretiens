



\paragraph{JR}

Bonjour Denise Pumain (\textbf{DP : } Bonjour), merci donc d'avoir bien voulu répondre à nos questions sur la théorie évolutive des villes. Juste pour fixer un peu le contexte, on propose un cadre de connaissances pour revisiter la démarche de la Géographie Théorique et Quantitative, et on pense que la théorie évolutive des villes, que vous avez développé maintenant depuis plus d'une vingtaine d'années, serait une illustration parfaite de ce cadre de connaissances. Je vous demanderai de développer certains aspects de l'histoire, et peut être de l'avenir de cette théorie évolutive. Pour commencer, une question très générale, mais pour situer les choses : qu'est ce que pour vous, vous appeleriez théorie évolutive, et est-ce que vous en avez une vision synthétique et unifiée aujourd'hui ?


\paragraph{DP}

Vous donnez d'emblée, à froid, une définition, hors contexte ! C'est une théorie géographique, en ce sens qu'elle s'intéresse à l'organisation des villes dans des territoires, à différentes échelles, et la principale ambition de cette théorie c'est de rassembler l'essentiel des faits connus sur ces objets en les plaçant non pas dans dans une théorie disciplinaire dans laquelle il y aurait des présupposés d'équilibre, de structures statiques arrêtées, mais dans une perspective spatio temporelle, qui permet de suivre ces objets sur de très longues périodes de temps, depuis leur émergence il y a 5000 ans, avec une attention d'une part aux principaux facteurs structurants de ces organisations, qu'on décrit à l'aide des termes de systèmes, et puis une attention également à signaler, observer les principales bifurcations qui marquent l'histoire de l'évolution de ces systèmes. Voilà, et c'est dans ce sens là qu'on imagine que les concepts des systèmes complexes peuvent nous aider à retrouver quelques propriétés des systèmes urbains. Essentiellement, considérer, du point de vue géographique les villes comme un systèmes dans les systèmes urbains.

\paragraph{JR} Donc là c'est Berry \ldots

\paragraph{DP}

C'est Berry mais considérablement enrichi et revisité, parce que pour nous , la ville est un objet complexe mais qui n'est pas un système entièrement autonome, chaque ville est insérée dans des réseaux, qui parfois sont très largement déterminants pour son futur devenir. D'où l'intérêt d'observer des grands ensembles de villes interconnectées, et qui co-évoluent, de façon à mieux comprendre les particularités de chaque ville.

\paragraph{JR}

Du coup, juste pour rebondir, et ça sera peut être une petite digression : donc on voit l'importance du cadre de la complexité et des Sciences des Systèmes Complexes dans cette théorie, du coup il y a bien l'idée de transcender les disciplines, car au début vous avez opposé à un point de vue disciplinaire, l'équilibre et tout ça, ça devait être en pensant à l'économie (\textbf{DP : } Tout à fait), l'économie mainstream qui ramène toujours tout à l'équilibre (\textbf{DP : }Exactement). Du coup est-ce qu'il y a aussi un géographie de l'équilibre, ou c'est complètement dépassé, c'était à l'époque de\ldots

\paragraph{DP}

Il y a eu quelques tentatives, notamment avec les cycles d'érosion, et l'idée géomorphologique, que on aboutissait nécessairement à une plaine, puis ce qu'on a appris de la tectonique a montré qu'on était là aussi dans des systèmes toujours en déséquilibre, contrairement à cet espèce d'arrêt stationnaire à un moment donné, qui n'est pas l'aboutissement nécessaire, je pense que socialement plus personne ne prétend qu'il y a équilibre, seulement c'est un instrument de pensée à partir duquel les économistes construisent leurs catégories, leurs raisonnements, hélas parfois leur modèles, pour les appliquer, alors que la recherche de cette optimisation ne peut être pour moi éventuellement qu'une construction de scenario, acceptables ou indésirables, qui peut tout au plus aider à penser certaines préconisations sur des durées très courtes, et en tout cas ça n'est jamais une construction suffisamment intéressante pour expliquer l'observation des évolutions.


\paragraph{JR}

Alors, mais en fait les, enfin c'est une impression que j'en ai - c'est encore une digression, je ne fais pas très bien les entretiens, mais ça me fait penser un peu à ça, et c'est important je pense - la plupart des tous ces travaux d'économistes et en fait aussi je pensais au transports, car j'ai fait un travail sur l'équilibre Static User Equilibrium en transports, ce gens là privilégient ces approches parce qu'ils veulent des systèmes qu'on peut résoudre analytiquement, qu'on peut maitriser, on peut trouver des solutions, et j'ai eu encore un débat avec mon ami qui est économiste l'autre jour, ils ont un modèle tout simple, et juste ils n'arrivent pas à le résoudre analytiquement, mais l'ordinateur trouve des solutions numériquement, mais ils ne veulent pas les accepter, ils veulent la résolution analytique, et ils sont bloqués depuis deux mois sur la résolution analytique, c'est assez incroyable, mais (\textbf{DP : } Certains mathématiciens, autrefois, en effet\ldots) quel est - alors c'est bien parce que ça nous fait rebondir sur l'histoire de co-évolution des outils et des méthodes avec la théorie, est-ce que la géographie a réussi à dépasser cette dépendance, euh, à un espèce d'idéal, je sais pas déjà d'où vient ce besoin d'avoir une résolution analytique, mais est-ce que la géographie a réussi à le dépasser, et si oui pour quelle raison, est-ce qu'il y a une particularité géographique qui fait qu'on est pas dépendant de ce besoin.

% 7min35
















