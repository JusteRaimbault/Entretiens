\documentclass[12pt]{article}

\usepackage[utf8]{inputenc}
\usepackage[T1]{fontenc}
\usepackage[margin=2cm]{geometry}

\title{Entretien avec Denise Pumain}
\date{31th March 2017}

\begin{document}

\maketitle

\paragraph{JR}

Bonjour Denise Pumain (\textbf{DP : } Bonjour), merci donc d'avoir bien voulu répondre à nos questions sur la théorie évolutive des villes. Juste pour fixer un peu le contexte, on propose un cadre de connaissances pour revisiter la démarche de la Géographie Théorique et Quantitative, et on pense que la théorie évolutive des villes, que vous avez développé maintenant depuis plus d'une vingtaine d'années, serait une illustration parfaite de ce cadre de connaissances. Je vous demanderai de développer certains aspects de l'histoire, et peut être de l'avenir de cette théorie évolutive. Pour commencer, une question très générale, mais pour situer les choses : qu'est ce que pour vous, vous apeleriez théorie évolutive, et est-ce que vous en avez une vision synthétique et unifiée aujourd'hui ?


\paragraph{DP}

Vous donnez d'emblée, à froid, une définition, hors contexte ! C'est une théorie géographique, en ce sens qu'elle s'intéresse à l'organisation des villes dans des territoires, à différentes échelles, et la principale ambition de cette théorie c'est de rassembler l'essentiel des faits connus sur ces objets en les plaçant non pas dans dans une théorie disciplinaire dans laquelle il y aurait des présupposés d'équilibre, de structures statiques arrêtées, mais dans une perspective spatio temporelle, qui permet de suivre ces objets sur de très longues périodes de temps, depuis leur émergence il y a 5000 ans, avec une attention d'une part aux principaux facteurs structurants de ces organisations, qu'on décrit à l'aide des termes de systèmes, et puis une attention également à signaler, observer les principales bifurcations qui marquent l'histoire de l'évolution de ces systèmes. Voilà, et c'est dans ce sens là qu'on imagine que les concepts des systèmes complexes peuvent nous aider à retrouver quelques propriétés des systèmes urbains. Essentiellement, considérer, du point de vue géographique les villes comme un systèmes dans les systèmes urbains.

\paragraph{JR} Donc là c'est Berry \ldots

\paragraph{DP}

C'est Berry mais considérablement enrichi et revisité, parce que pour nous , la ville est un objet complexe mais qui n'est pas un système entièrement autonome, chaque ville est insérée dans des réseaux, qui parfois sont très largement déterminants pour son futur devenir. D'où l'intérêt d'observer des grands ensembles de villes interconnectées, et qui co-évoluent, de façon à mieux comprendre les particularités de chaque ville.

\paragraph{JR}

Du coup, juste pour rebondir, et ça sera peut être une petite digression : donc on voit l'importance du cadre de la complexité et des Sciences des Systèmes Complexes dans cette théorie, du coup il y a bien l'idée de transcender les disciplines, car au début vous avez opposé à un point de vue disciplinaire, l'équilibre et tout ça, ça devait être en pensant à l'économie (\textbf{DP : } Tout à fait), l'économie mainstream qui ramène toujours tout à l'équilibre (\textbf{DP : }Exactement). Du coup est-ce qu'il y a aussi un géographie de l'équilibre, ou c'est complètement dépassé, c'était à l'époque de\ldots

\paragraph{DP}

Il y a eu quelques tentatives, notamment avec les cycles d'érosion, et l'idée géomorphologique, que on aboutissait nécessairement à une plaine, puis ce qu'on a appris de la tectonique a montré qu'on était là aussi dans des systèmes toujours en déséquilibre, contrairement à cet espèce d'arrêt stationnaire à un moment donné, qui n'est pas l'aboutissement nécessaire, je pense que socialement plus personne ne prétend qu'il y a équilibre, seulement c'est un instrument de pensée à partir duquel les économistes construisent leurs catégories, leurs raisonnements, hélas parfois leur modèles, pour les appliquer, alors que la recherche de cette optimisation ne peut être pour moi éventuellement qu'une construction de scenario, acceptables ou indésirables, qui peut tout au plus aider à penser certaines préconisations sur des durées très courtes, et en tout cas ça n'est jamais une construction suffisamment intéressante pour expliquer l'observation des évolutions.


\paragraph{JR}

Alors, mais en fait les, enfin c'est une impression que j'en ai - c'est encore une digression, je ne fais pas très bien les entretiens, mais ça me fait penser un peu à ça, et c'est important je pense - la plupart des tous ces travaux d'économistes et en fait aussi je pensais au transports, car j'ai fait un travail sur l'équilibre Static User Equilibrium en transports, ce gens là privilégient ces approches parce qu'ils veulent des systèmes qu'on peut résoudre analytiquement, qu'on peut maitriser, on peut trouver des solutions, et j'ai eu encore un débat avec mon ami qui est économiste l'autre jour, ils ont un modèle tout simple, et juste ils n'arrivent pas à le résoudre analytiquement, mais l'ordinateur trouve des solutions numériquement, mais ils ne veulent pas les accepter, ils veulent la résolution analytique, et ils sont bloqués depuis deux mois sur la résolution analytique, c'est assez incroyable, mais (\textbf{DP : } Certains mathématiciens, autrefois, en effet\ldots) quel est - alors c'est bien parce que ça nous fait rebondir sur l'histoire de co-évolution des outils et des méthodes avec la théorie, est-ce que la géographie a réussi à dépasser cette dépendance, euh, à un espèce d'idéal, je sais pas déjà d'où vient ce besoin d'avoir une résolution analytique, mais est-ce que la géographie a réussi à le dépasser, et si oui pour quelle raison, est-ce qu'il y a une particularité géographique qui fait qu'on est pas dépendant de ce besoin.

% 7min35


\paragraph{DP}

Je pense que la théorie économique mainstream, enfin ça remonte à Walras a un certain mécanicisme, en même temps que une volonté de pouvoir formaliser par les mathématiques la partie la plus facile à numériser du contenu social, qui est sa dimension comptable, laquelle était déjà établie empiriquement depuis très longtemps, par les comptabilités à double entrée des entreprises qui ont été mises au point plusieurs siècles avant, donc y'a tout ce passé, plus les intérêts que vous signalez actuellement pour disposer de résultats tangibles sous forme de calcul, et d'évaluations de coûts, je ne suis pas en train de dire que l'économie n'a pas sa raison d'être quelque part pour un grand nombre d'opérations, dont certaines éventuellement touchent les villes. Alors nous cette approche je ne pense pas qu'on puisse dire que nous l'ayons dépassée, dans la mesure où elle ne nous a jamais rendu de très grands services, parce que pendant très longtemps la géographie a été plutôt littéraire et descriptive, et quand elle est passée au stade de l'analyse quantitative, elle avait déjà défini des objectifs de recherche qui admettaient les inégalités, la diversité comme un objet non pas à récuser non pas à réfuter non pas à reléguer dans un petit résidu de modèle, mais au contraire comme une substance qui faisait l'intérêt. Même un géomorphologue très quantitativiste, Pierre Birot parlait de la beauté de ce que jamais l'on ne verra deux fois. Donc l'unicité des objets géographiques, qui a fait le plaisir des grandes découvertes, des récits de voyage, qui aujourd'hui fait le bonheur de beaucoup de touristes, c'est quelque chose qui demeure le-un des stimulants ludiques de la géographie dans la recherche, et aussi un des ressorts de sa recherche d'explications et de formulation de théories. En ce sens là, on peut comprendre les réactions négatives de certains géographes à l'encontre des premiers essais de formalisation mathématique en géographie, le plus drôle c'est Walter Christaller inventeur de la théorie des lieux centraux, se moquant de Zipf ou Auerbach, en disant que la loi rang-taille n'était qu'un simple jeu de nombres, c'est amusant car Christaller s'est fait critiqué de son côté pour être, pour donner un schéma très caricatural de l'organisation des - enfin jugé comme caricatural par certains - de l'organisation des villes, et Pierre Georges encore dans les années 1950 ou 60, récusait l'idée qu'on ai besoin de faire une géographie des supermarchés. Donc la dimension comptable et analytique en géographie, elle a pu être contestée, et elle a été récusée, souvent vue comme un danger, précisément de contraindre ou d'enfermer les territoires dans des fonctionnements qui auraient été trop normatifs, trop homogènes, mais la géographie ne s'est jamais donné pour objectif de proposer quelque chose d'aussi fort que le modèle du marché offre et demande, enfin d'identifier la variable-les variables clés qui devraient rendre compte d'un ensemble de faits suffisants.

\paragraph{JR}

Donc jamais de loi universelle, mais identifier des processus qui reviennent plus ou moins régulièrement, ...

\paragraph{DP}

On parlait de régularité plutôt que de loi, c'est une discussion que nous avons encore avec Michael Storper qui refuse absolument l'idée qu'il y ait des lois, je pense qu'il oublie que le mot loi c'est un emprunt des sciences dures au monde social, puisque les lois sont énoncées par des puissances, mais blague à part, je pense que les, ces régularités elle sont, comme toujours en sciences humaines et sociales, elles sont nécessairement contextualisées, et celles qu'on est capable d'énoncer à un niveau si élevé d'abstraction qu'elles sont vraiment universelles, elles sont très rares, et quand on les énonce, c'est sous la forme verbo-conceptuelle, à la Tobler, la première loi de la géographie qu'est ce que c'est : tout interagit avec tout, mais deux choses proches ont plus de chances d'interagir que deux choses éloignées, tel quel l'énoncé n'est pas, n'est qu'en partie falsifiable, c'est déjà un énoncé falsifiable, donc c'est vraiment une loi, une théorie au sens très fort, pour la géographie, bon et ensuite on se contente de lois bien mieux contextualisées, je crois.

\paragraph{JR}

Parfait. Alors je propose qu'on revienne plus en détails sur l'histoire de la théorie évolutive, et son contexte. Alors je propose une grille de lecture peut être un peu simple, mais temporelle en fait, pour clarifier un peu certaines, certains points qui auraient pu être particuliers, ou certains moments clés. Alors voilà, là j'ai essayé de faire, c'est pas fini donc, mais une, je vous montre l'image, un peu une synthèse, c'est pas complet du tout parce que pour l'instant j'ai pris les références que j'avais dans ma biblio à moi, et je sais que j'ai plein de papiers à ajouter là dedans, mais donc dans le temps, et dans les domaines de connaissance, essayé d'insérer les publications, et soit les notions, soit les modèles, soit les bases de données, donc en rouge c'est plus des notions, bon c'est pas unifié, je referai, et aussi des personnes, alors j'ai pas du tout l'aspect social pour l'instant, mais les noms des personnes dans les publications, mais c'est vrai qu'on pourrait mettre d'autres entités qui sont les personnes en fait, où elles ont évolué tout ça, euh voilà...

\paragraph{DP}

Vous voulez que je commence par le commencement ?


\paragraph{JR}

Oui du coup, alors ça c'était juste pour peut être un peu situer les choses, et je suis pas sûr aussi d'avoir commencé au bon, est-ce que c'est l'article séminal ou pas, donc voilà mon premier point c'était les origines, quelles auraient été les précurseurs, vous avez déjà donné un peu la réponse au début en parlant des théories des systèmes complexes tout ça, des précurseurs, est-ce qu'il y a eu un article ou des actes fondateurs, un moment clé où on s'est dit maintenant on a une théorie évolutive, est-ce que le fameux (Pumain, 97) c'est le...

\paragraph{DP}

Le Pumain 97 c'est l'article séminal (\textbf{JR} d'accord) qui enregistre l'expression de la théorie évolutive, mais je pense qu'on peut, enfin, remonter beaucoup plus loin, sur mon parcours personnel, mais je vais vous le faire d'une manière qui correspond bien à votre démarche intégrée, parce que je trouve ça amusant, je l'ai jamais fait encore, mais bien entendu les précurseurs ils sont très nombreux, dans l'histoire des idées et des travaux sur les villes, donc là un jour je vous referai tout l'état de la question, mais ça pourrait être très long, trop long sûrement. Mais ce qui est amusant, c'est que c'est un peu par accident que j'ai commencé à travailler sur les villes, puisque avec Marie-Claire Robic, et à l'instigation de Philippe Pinchemel, avec qui nous souhaitions faire notre mémoire de maitrise en 1968, ça a commencé par une question de données, une géographe indienne de passage avait découvert à l'INSEE des tableaux d'information tirés des recensements qui avaient été construits en réponse à une question posée aux résidents : où habitiez vous au recensement précédent ? Ca couvrait la période 1954-1962, et ça donnait une description assez précise de l'âge et des lieux d'où venaient les populations migrantes, à l'intérieur de la France, et c'était distribué assez précisément en terme de localisation géographique, et intelligemment parce que ça avait été construit autour des agglomérations urbaines, déjà en dépassant l'échelle communale, et pour une agglomération donnée de plus de 50000 habitants, on pouvait identifier l'origine des migrants, des communes rurales du même département, de tel ou tel autre département, etc., bon on a réaggrégé par la suite, et donc on a pu enregistrer les effectifs de toutes les personnes qui s'étaient déplacées entre deux recensements, et avec pour point d'arrivée ou d'origine nos agglomérations de plus de 50000 habitants. Et là dessus donc, l'intérêt pour les villes a émergé d'une disponibilité de données pour en parler, pour travailler dessus vec quelque chose qui d'emblée allait, parlait de changement, puisque c'était une population qui se déplaçait, d'emblée on a été confrontés aux outils et aux méthodes, les outils une machine à calculer pour faire les multiplications et les divisions avec un bruit de moulin à café, je me demande pas si on le retrouve maintenant dans les cafés branchés, un bruit rrr de moteur. Une toute petite, comment on va dire, sensibilisation statistique qui nous permettait de deviner que moyenne et écart-type avaient de l'intérêt pour étudier les distributions, et une énorme frustration méthodologique, parce que moi j'avais des profils de migration, en fonction de l'origine plus ou moins lointaine, et Marie-Claire qui travaillait sur les structures démographiques avait des profils d'âge, comment on fait pour analyser un profil dans son ensemble, et pas seulement des variables deux à deux, avec un coefficient de correlation. Et ça à l'époque, on était quand même, rétrospectivement j'étais fâchée que personne ne nous dise : voilà les multivariés ça existe, les ordinateurs ça commence à arriver, on en a besoin... Donc cette frustration ; ensuite on a prolongé le travail, donc réussi avec un mémoire de maitrise rédigé en commun, une innovation pour l'époque, puis passé le concours de l'aggregation, donc une interruption, mais quand on est revenues ensuite comme assistantes aggrégées, dans une université, on avait, moi j'avais appris le fortran entre temps, donc déjà on avait accès à un centre de calcul, et on a eu à répondre à un appel d'offres sur la croissance des villes, qu'est ce qui explique que telle ou telle ville croisse et d'autres pas. Et, tellement le sujet, enfin la science sociale était, on va dire, éventuellement politisée à l'époque, en tout cas sensible à ces démarches, la question c'était est-ce que la couleur du maire, la couleur politique du maire, a une influence sur ces processus. Avec l'idée qu'il pourrait y avoir des personnes plus ou moins favorables à ceci. Et en se penchant sur, en dialoguant avec les personnes en question qui faisaient des statistiques aussi, il y avait une contradiction flagrante entre ce qu'on nous disait ``oui nous avons fait tel programme de construction de logements, oui nous avons été favorables à l'implantation de telle ou telle entreprise, et puis ce qu'on lisait dans les statistiques, qui était qu'on voyait la même chose partout. D'où l'intérêt déjà pour les régularités - pour observer des régularités, ne pas se contenter d'une explication locale, et donc assez vite, je sais plus sous quel déclencheur, en tout cas on est passé à la loi rang-taille et au modèle de Gibrat, sur lequel j'ai travaillé, la dynamique des villes, c'est 1982. Et entre temps, j'avais publié avec Thérèse Saint-Julien, en 1978, les dimensions du changement urbain, et là quand même la plus grosse découverte c'était la co-évolution socio-economique des villes. C'était (\textbf{JR} Donc c'était déjà dans le 1978...) - ah oui oui - (\textbf{JR} Donc la théorie évolutive était déjà implicitement) - y'avait un germe. Déjà une théorie c'est jamais une affaire solitaire (\textbf{JR} oui bien sûr), c'est jamais l'invention d'une personne, et puis c'est quelque chose qui mûrit, forcément un peu lentement, mais de manière complètement intriquée entre données, modèles, observation, publications\ldots Enfin voyez, et vie de laboratoire, échanges extérieurs, puisque chemin faisant, on est allées voir des gens qui travaillaient avec la théorie de l'information, je suis allée voir, je suis allée écouter un jour un exposé sur l'entropie par Peter Allen, plus tard, et je participais aux réunions des géographes quantitativistes qui se formaient aux disciplines des statistiques, des probabilités, enfin toutes choses qui tombaient sous le sens et qui devenaient accessibles à l'époque. Donc c'était un propos liminaire, mais pour vous montrer à quel point cotre idée, quand j'ai lu géographie intégrée, je me suis dit mais pourquoi, bon sang mais bien sûr - non c'est, vraiment je suis très contente que on en vienne aujourd'hui à identifier une démarche qui en fait nous a inspirée, sans que nous l'ayons formulée en tant que telle dès le départ. (\textbf{JR} Une idée mûrit\ldots ) Oui et en même temps, ça nous paraissait nécessaire, alors moi j'ai fait une formation mathelem au baccalauréat, donc les professeurs géographes qui proposaient des énoncés testables, vérifiables, ou partiellement quantifiés, ils m'attiraient ; voir que l'on pouvait - je me souviens d'un cours de Michel Rochefort, qui par ailleurs n'était pas très nourri, mais qui néanmoins nous avait dit, en s'appuyant sur d'autres travaux, ça ça devait être je crois en 67, en cours de licence ; il nous disait bah voilà, dans une ville, on peut avoir jusqu'à 75\% d'emplois industriels mais pas plus, dans les années 50, on est toujours dans cette période % 23min15
où y'avait encore des villes minières, des villes sidérurgiques, et là je me dis ah bon, on peut donc mesurer quelque chose qui parle de l'activité des villes, et donc très vite avec Thérèse on a cherché des données qu'on a trouvé, dans les recensements, pour exploiter les profils économiques, sur lesquels on a écrit donc ce premier ouvrage, 78, qui à la fois faisait un peu la nique au, puisqu'on était chacune dans deux laboratoires différents avec deux directeurs de thèse de doctorat différents, et on osait publier un livre avant d'avoir soutenu notre thèse, c'était un scandale, bref... la dimension sociologique du défi, du challenge, n'est pas complètement absente.

\paragraph{JR}

Oui du coup je rebondis sur le côté du labo, j'avais noté cette question, quel était le rôle des structures de recherche, et particulièrement de Géocités, enfin Géocités est un peu né en relation avec toute cette aventure ? (\textbf{DP} alors...), où il n'y a aucun rapport ?

\paragraph{DP}

Elle s'est construite, si y'a un rapport, en ce sens où elle s'est construite autour de trois personnes, Thérèse Saint-Julien et moi, en même temps Marie-Claire Robic qui était dans notre groupe de travail a été nommée à Créteil, et puis avait choisi de faire de l'histoire des sciences, elle s'était un petit peu éloignée de ces travaux, mais à trois on avait, nous faisions des réunions de bibliographie, de réunions de..., et j'avais entamé un enseignement de méthodes quantitatives dès 72, auquel Thérèse a du se raccrocher vers 75, et je pense que c'est en 76 qu'on est venues ici partager un bureau, avec un professeur d'informatique qui avait été nommé à Paris 1, pour accompagner le mouvement d'enseignement des statistiques dans, pour les sciences humaines et sociales. Et voilà, donc en fait, construire un laboratoire c'était aussi bien une question d'organisation matérielle, puisque les assistants et maîtres assistants à Paris 1 avaient été gratifiés d'un petit bureau de 8 ou 9m$^2$, au quatrième étage, qui devait servir à 40 personnes, et le travail collectif y a démarré parce qu'on s'y retrouvait, Thérèse et moi, tous les matins à 9h, aussi bien pour préparer par avance les cours, ou faire les travaux d'analyse quantitative, et la rédaction d'articles et d'ouvrages, et donc on avait compris, on a fait installer une armoire, on avait compris l'intérêt d'une organisation matérielle, et collective du travail, donc c'est ça qui nous a motivées pour créer ce qui s'appelait d'abord une jeune équipe, et là c'est vraiment le CNRS qui a lancé des programmes permettant l'accrochage institutionnel d'initiatives intellectuelles, et donc ce laboratoire est devenu, on appelait ça d'abord unité associée au CNRS, je crois que ça s'est produit sous l'appellation équipe Paris, c'était en 1984 peut-être, et puis en 92, on, le CNRS nous a demandé de nous regrouper avec EHGO, et c'est là où l'appellation Géographie-cités a été choisie (\textbf{JR} d'accord, EHGO existait déjà par ailleurs\ldots ) ; EHGO existait déjà par ailleurs, avait été fondé je crois dès 76 par Philippe Pinchemel et M. Maulat, qui était un historien, comme un laboratoire de géographie historique et d'histoire de la géographie, qui avait des locaux rue Maller, des locaux de l'université Paris 1, qui était également associé au CNRS, peut être avant l'équipe Paris, et pour des raisons de réduction du nombre d'équipes en SHS, c'est un sujet qui revient régulièrement au CNRS, et puis parce que nos méthodes de travail n'étaient pas si éloignées finalement, on nous a demandé de nous regrouper, et EHGO est venu nous rejoindre rue du Four. Voilà. Entre temps c'est Thérèse Saint-Julien qui avait dirigé l'équipe Paris quand elle a été créée.

\paragraph{JR}

D'accord. Et alors du coup, c'est un... le livre de Lena Sanders sur Synergétique, c'est en 1992 il me semble (\textbf{DP} Oui). Alors quel est le lien avec la théorie évolutive donc ?


\paragraph{DP}

Alors dans les recherches, dans cette découverte de coévolution des villes, qu'on appelait pas encore coévolution, mais de parallélisme, et de transformation des villes en termes d'image de marque et de modernité technique, qu'on avait acté 78, on a cherché des formalisations d'abord mathématiques pour essayer de rendre compte de ce changement, qu'on observait avec des analyses permettant de, des analyses multivariées, permettant de simuler des traj - enfin pas de simuler, mais de construire des trajectoires à partir de quatre photographies transversales, c'est à dire à des moments différents (\textbf{JR} une typologie des trajectoires\ldots ) - voilà. Et ça c'était quelque chose d'un peu insatisfaisant, puisque on avait pas le processus qui conduisait à la transformation. Donc à la recherche de ça, et après avoir entendu Peter Allen dans un exposé à Créteil sur l'entropie, mais qui parlait de modalités de transformation de ces systèmes auto-organisés, en terme d'ordre par fluctuations, beaucoup de petites transformations aléatoires par rapport à la structure qui ne la transformaient pas, mais certaines agissant de façon répétée contribuaient, et ça nous on avait eu l'intuition de cette forme de changement, en observant l'émergence d'un deuxième facteur dans les analyses factorielles, et en comparant l'évolution des trajectoires, enfin ça se faisait à la main et de manière très ad hoc vous me direz, donc Peter Allen proposait des modèles pour analyser ces transformations, mais qui allaient, qui étaient des modèles plutôt intra-urbains ou intra-régionaux. Donc on a accepté de faire le détour, notre premier détour, pour tester la manipulation de ces systèmes d'équations non-linéaires, permettant éventuellement de rendre compte des changements observés. Ca nous a obligé à changer d'échelle, à quitter le système de villes pour passer dans les villes, puisque, et encore on était pas très satisfait, puisque du point de vue statistique, on pouvait seulement pour une agglomération urbaine, avoir l'échelle communale, avec des données socio-économique suffisantes pour renseigner le modèle, donc sur l'agglomération de Rouen on était à quelque chose comme 17 communes, c'était, mais c'était un bon point de départ, sur lequel Lena Sanders a fait sa thèse, et sur lequel on s'est ensemble bien cassé les dents, alors Lena Sanders je l'ai rencontrée en 1982, à San Miniato. San Miniato, c'était une première énorme rencontre de gens qui aujourd'hui, dans tous les domaines de l'observation des villes et des territoires, et en modélisation, sont encore là, y'avait Peter Mallcamp, y'avait Giovanni Rabino qui vient de disparaître, y'avait Regapi, y'avait Ora Regiani, y'avait, Gunther Haag était là, y'avait des américains Dan Griffiths, Leslie Corie était là, on peut avoir des listes complètes, mais c'était vraiment un noeud de rencontre intéressant, et c'est là où j'ai compris que l'Europe était en avance par rapport aux américains, qui étaient tous complètement décontenancés par la présentation de Gunther Haag et Wolfgang Weidlich sur les systèmes auto-organisés, alors que moi j'avais été déjà bien en contact avec donc l'équipe de Prigogine, via la rencontre avec Peter Allen, Michel Sanglier, Prigogine lui-même, via des colloques de dynamique des systèmes à Boston et à Bruxelles. (\textbf{JR} Oui les américains étaient plus dans les stats spatiales\ldots ) - et plus dans le modèle économétrique finalement (\textbf{JR} oui oui oui). Hors Lena Sanders a d'emblée, était aussi intéressée par ça, et elle a bien voulu faire une thèse sur l'application de ce modèle de Peter Allen, donc au début on a ramé beaucoup car on nous avait donné une mauvaise version Fortran du modèle, qui ne pouvait pas marcher mathématiquement, et à force de travailler dessus jusqu'à minuit au centre de calcul au Panthéon, et de retourner à Bruxelles en montrant qu'on avait fait toutes ces expériences qui marchaient pas - ah oui, parce qu'on nous disait au téléphone il faut chipoter, (rires) Michel Sanglier dixit, avec les paramètres etcaetera, bref et découvrant le modèle, ah oui on vous avait donné, mais effectivement cette équation-là\ldots , bon bref, on est revenues avec une nouvelle version - \textbf{JR} c'est amusant parce que c'est la même chose aujourd'hui \textbf{DP} c'est vrai ? \textbf{JR} bah faut touiller les paramètres un peu, ça marche pas ça marche pas, ah oui au fait tout était faux depuis le début, je pense qu'on a les mêmes problématiques qui ressortent - \textbf{DP} Et c'est quelles, des erreurs qui vous avaient été sciaemment données, en connaissance de cause, ou ? - \textbf{JR} bah des fois y'a des biais dans les algorithmes et on le sait pas, puis trois jours après y'a une correction du bug dane le truc, après c'est nous qui avons mal codé une équation\ldots - \textbf{DP} En tout cas ça peut arriver aussi oui. Mais là moi j'étais assez naïve en matière d'équations non-linéaires, j'avais refait un peu de formation mathématique quand même, mais en particulier pour comprendre tout ce qui se disait sur la déduction de la loi rang-taille en terme de maximisation d'entropie, qui est complètement opaque et mensonger à la limite, mensonge par omission, bref. Donc, mais comprendre, enfin anticiper le fonctionnement dynamique d'une équation juste en la regardant, d'autant que comme je l'ai expliqué à Fabrizzio justement hier, qui me disait, mais on rejoint, la formalisation fait se rejoindre la réalité et la modélisation, il me disait mais quelle formalisation, car en fait y'a aucune ambiguïté effectivement si on s'en tient à un formalisme mathématique, ou même algorithmique, en revanche, dès l'instant qu'il est renseigné par des concepts, et des contenus mis sous des paramètres, si ça ne se comporte pas comme, conformément à la signification qu'on y a mise, on a un problème, donc y'a pas que l'implémentation mathématique et informatique de la formalisation, y'a aussi toute une formalisation conceptuelle qui est présente dans le modèle mathématique, manipulé par des - \textbf{JR} une sorte de validation externe en fait ? enfin si on a un paramètre et qu'on lui donne une plage de valeurs où il a le droit d'aller, et - \textbf{DP} oui, et qu'il y obéit pas, ça fait un souci pour le modélisateur de sciences humaines, nécessairement. Et je n'ai compris ça, enfin j'ai identifié le problème très clairement que avec une stagiaire belge qui est venue ici, des années plus tard, et avec laquelle on, qui programmait et qui avait programmé en termes d'équations non-linéaires, avec le modèle de Volterra-Lotka, et on avait décidé de l'appliquer aux échanges centre-périphérie, on a travaillé avec Lena là-dessus, entre une agglomération et sa périphérie, le coeur de l'agglomération absorbant d'abord les communes rurales en périphérie, puis redéversant sa population ; on se disait c'est facile à programmer avec, on connait les modalités qui sont pareille partout ou presque, ou on les fait fonction de la structure d'âge de la population, on a des mouvements migratoires, ça doit être facile. Et en fait on s'est cassé les dents, on arrivait pas à faire en sorte que le x-y, enfin le paramètre d'interaction du Volterra-Lotka, ne contienne que des migrations, que le natalité-mortalité ne contienne que la variation naturelle de la population, c'était in-fai-sable. Donc\ldots - \textbf{JR} En fait il doit y avoir plusieurs solutions équivalentes\ldots \textbf{DP} Probablement, mais en fait il aurait fallu une décomposition multi-agents en fait (\textbf{JR} oui c'est ça en fait) - un décomposition par sous-ensembles, et contraindre chaque sous-ensemble à fonctionner - \textbf{JR} Car sur le modèle agrégé on pourra le calibrer de plein de façons différentes en fait, et on saura pas laquelle est la vraie, enfin celle où l'interprétation\ldots - \textbf{DP} Voilà, on peut pas maîtriser le, ça c'était une belle leçon quand même par rapport aux modèles mathématiques, qui évidemment sont plus propres, plus beaux, plus économes, plus faciles à transmettre, tout ce qu'on veut comme avantages certains\ldots - \textbf{JR} Y'a une publication sur le Lotka-Volterra ? Car je trouve ça assez intéressant - \textbf{DP} Y'a un rapport, donc il doit être ici, j'espère l'avoir gardé quelque part dans la bibliothèque ; j'arrive plus à retrouver son nom. \textbf{JR} Parce que c'est un sentiment que j'ai, aujourd'hui les, quand je parle un peu avec tout le monde et tout, tout le monde a un peu perdu le côté systèmes dynamiques, alors que pourtant il faut toujours l'avoir à l'esprit, que dans certains cas ça peut marcher, et du coup je sais pas, faudrait regarder si, du coup vous l'aviez fait à une époque mais ça a été oublié\ldots \textbf{DP} Alors j'ai, ah oui on a pas recommencé\ldots \textbf{JR} Parce que quand je discute avec les gens qui font du multi-agent, ils ont pas en tête que dans certains cas, en effet un système dynamique ça serait peut être mieux, ou que dans ces cas là simples on peut agréger pour faire un système dynamique, ou là c'est pas un multi-agent mais c'est un système dynamique, on peut généraliser\ldots \textbf{DP} Pour les proies-prédateurs, y'a pas mal de choses de faites quand même\ldots \textbf{JP} Par des Géographes ? \textbf{DP} Des géographes qui travaillent sur, oui dans les Causses, sur les moutons et loups\ldots \textbf{JR} Ah oui oui, de la biogéographie, mais du coup pas appliqué aux villes\ldots \textbf{DP} Ah non, aux villes\ldots \textbf{JR} Aux villes, dynamique des villes\ldots \textbf{DP} Alors aux villes, les systèmes avec équations à la Forrester, y'a eu une application sur Carpentras, faite par un élève de\ldots, un élève d'un économiste, dont le nom ne me reviendra pas, mais bon ça peut se retrouver, mais qui a, et puis par une fille qui s'appelait Christine Alexandre qui avait essayé de faire ça à Toulouse, bref y'a trois petites tentatives comme ça, mais qui, alors on travaillait avec Dynamo, et écrire en Dynamo des interactions spatiales entre un centre et une ou des périphéries, apparemment c'était carrément dur - \textbf{JR} Donc là on revient au outils en fait, au problème de l'outil, il y aurait fallu\ldots Mais peut être que revisiter des choses qui ont été faites dans le passé, avec les nouveaux outils, avec les nouvelles méthodes, très intéressant. D'ailleurs du coup un aparté - en fait j'avais, en fait on pourrait faire ça de manière complètement systématique, en notant toutes les personnes, toutes les références, toutes les notions, et le coder en graphe, et faire un graphe multi-layer (\textbf{DP} oui), et y'aurait peut être des choses qui sortiraient de l'analyse, une analyse un peu quanti de, enfin du coup pas que quanti mais - \textbf{DP} De toute cette bibliographie\ldots \textbf{JR} Voilà en fait, et quantifier la co-évolution, des fois on dit ah bah dans cet exemple là, l'outil a plus servi, du coup % 40min 34
on apprend à coder des liens de causalité - \textbf{DP} oui moi ça m'a parlé votre affaire, parce que effectivement, d'expérience ça ne peut que marcher quoi - \textbf{JR} Du coup on garde ça en tête pour le futur, refaire ça de manière codée, et faire des analyses de graphe, parce que y'a plein de façons de faire des liens entre tout ça, on peut faire des liens de citations, on peut faire des liens de dépendance à un outil, de dépendance à une truc théorique, les liens entre concepts, les liens entre personnes les liens sociaux, rajouter le labo, les liens géographiques, ça peut être génial - \textbf{DP} ah oui oui oui tout à fait (rires) \ldots Donc Lena a fait sa thèse sur le modèle de Peter Allen et a présenté donc un article, que je viens de mettre sur hal d'ailleurs, c'est des vieux papiers qui n'avaient jamais été mis en dépôt, et ensuite on a retravaillé ensemble, ça c'est, on s'est rencontrées en 82, en 84 y'a eu un autre, la suite du colloque organisé par l'Otan - Transformations [?] à [?] au Danemark, et là Gunther Haag y était également, et avait commencé à travailler avec le modèle synergétique, et sur des questions de migration, et moi j'avais pour un usage pédagogique, les bases de données sur les migrations interrégionales à plusieurs dates en France, donc on lui a proposé ça, il a dit ok, on a commencé à travailler, j'ai fait le chapitre sur la France, dans un bouquin multi-pays, à 11 pays, donc appliquant le modèle de la synergétique aux migrations interrégionales, et Lena parallèlement, a commencé à appliquer une variante de ce modèle aux villes, dont elle a fait son bouquin, 92, sur Systèmes de villes et synergétique. Donc on avait comme ça expérimenté deux types de modèles d'équations différentielles non-linéaires, avec à chaque fois des satisfactions, des insatisfactions, et parallèlement, France Guérin, qui se trouvait être la belle-soeur de Jacques Werber - \textbf{JR} Ah donc le côté sociologique\ldots - \textbf{DP} le côté sociologique oui ! Et France Guérin a fait sa thèse avec moi, sur la loi rang-taille, système d'évolution des villes, elle a prolongé en fait la dynamique des villes jusque en 1975, moi j'avais fait ça sur le 19ème siècle à partir de la base de données que j'avais, enfin bref, donc elle a fait aussi un peu d'analyse spatiale sur ces résultats là, et donc elle nous emmenées à Paris 1, une dame faisait à la fois des réseaux de neurones, et puis s'intéressait aussi au multi-agent, sensibilisation, accueil ici d'un étudiant donc Stephane Bura, qui a fait le premier Simpop - \textbf{JR} D'accord - \textbf{DP} Voilà\ldots - \textbf{JR} Alors du coup le premier Simpop c'est bien 97, c'est ça ? - \textbf{DP} Oui dans ces eaux là, ça doit être sur le site de Simpop, que Paul Chapron a remis en ordre, c'est spécifié - \textbf{JR} c'est le papier ``A multi-agent system for the study of urbanism'' ? - \textbf{DP} alors le papier c'est peut être même 96, publication dans Geographical Analysis - \textbf{JR} Ah du coup c'est peut être pas le même - \textbf{DP} Avec Stéphane Bura - \textbf{JR} Y'a Lena Sanders, vous, Stéphane Bura, et quelqu'un d'autre je crois - \textbf{DP} Le premier auteur c'est Bura, puis après il doit y avoir oui Sanders, Pumain, qui est ce qu'il peut y avoir encore\ldots 

\paragraph{JR}

Et du coup, là comme en 97 y'avait aussi l'article séminal de la théorie, on était vraiment dans une symbiose de la modélisation et de la théorie à ce moment là \ldots

\paragraph{DP}

Oui oui, et avec l'impression d'avoir trouvé, avec le multi-agents, l'outil suffisamment flexible pour permettre d'intégrer de manière plus, beaucoup plus souple, enfin qu'avec les équations, les interactions spatiales, différenciées suivant les fonctions urbaines, puisque c'était ça qui nous importait, d'avoir des places centrales et des villes spécialisées, industrie, tourisme, en première base, dans la mesure où oute la dynamique des villes sur le plan socio-économique est quand même emportée par leur activité principale, alors qui conditionne la sociologie des catégories socio-professionelles, les niveaux de formation. \textbf{JR} Du coup les agents hétérogènes c'est idéal\ldots - \textbf{DP} ah oui absolument nécessaire. Et d'autant qu'on a compris depuis, alors en retravaillant dessus, et aussi avec les lois d'échelle par exemple, à quel point c'était constitutif, cette espèce de noria d'exploitation de la force de la division interurbaine du travail dans le système des villes, avec relocalisation des activités obsolètes dans les petites villes moins chères - \textbf{JR} ah c'est la thèse d'Olivier, tout ça - \textbf{DP} Oui, c'est une mise à l'épreuve d'une des\ldots 

\paragraph{JR}

D'accord. Oui du coup on y est déjà un peu passé je pense, mais un deuxième point après les origines c'était la genèse, alors du coup est-ce que vous ressentez des points clés - j'ai mis des milestones, je sais pas comment on dit en français - particuliers, et peut être ventilés par domaines, bah là y'a eu ce point de bifurcation sur le modeling, par exemple Simpop est apparu, et après\ldots Ou sur la théorie ou sur les outils, du coup sans aller peut être vraiment jusqu'au présent parce que, je sais pas si on peut mettre une date limite, parce que là on peut être pas assez de recul encore sur depuis 2013-14\ldots 

\paragraph{DP}

Enfin, une grosse bifurcations, bon, déjà avec Simpop2 et Benoît Glisse, qui nous a permis de, et puis le travail de Anne Bretagnolle sur les bases de données Europe Etats-Unis, on a pu, comment dire - \textbf{JR} Commencer à tester ? - \textbf{DP} Voilà, changer d'échelle, en terme de, on a pu, caler empiriquement le modèle. Alors Lena et Hélène surtout, j'ai un peu participé, l'avaient déjà fait sur Eurosim, qui a une publication dans Cybergeo, sur Eurosim - \textbf{JR} C'est 2000 ça Eurosim - \textbf{DP} Oui, donc avec un calage sur les données empiriques, la publication faite avec Anne Bretagnolle sur Simpop2 c'est 2010, parce que ça a pris beaucoup de temps aussi, et avec la comparaison Europe Etats-Unis, on a appris d'avantage de, et vu aussi quelles pouvaient être les limites nouvelles en terme d'expérimentation à la main - \textbf{JR} C'est là qu'on commençait à parler des problématiques de exploration systématique, et les première idées de pourquoi une plateforme intégrée, et\ldots \textbf{DP} Et voilà, et ça c'est Geodivercity qui nous a donnée les moyens matériels et personnels donc, et c'est avec des sous qu'on peut faire travailler les gens, de vraiment changer de régime de collaboration avec les informaticiens, c'est à dire au lieu que le modèle soit fait séparément, ou en interaction, mais implémenté séparément je veux dire, par un informaticien à partir de ce qu'il comprenait des régularités que nous souhaitions mettre, faire apparaître dans le modèle, ça ça été jusqu'à Benoît Glisse et Simpop2, et ensuite y'a Thomas Louail qui a fait son propre modèle, qui donc maitrisait d'avantage l'interaction géographie-informatique, et c'est surtout avec Geodivercity et Clémentine Cottineau donc, 2014, Marius, que là vraiment y'a eu un co-travail en co, en intégration complète des intérêts de la recherche pour trouver des procédures de validation qui, et un saut qualitatif et quantitatif, en passant d'une centaine de simulations manuelles, à 500 millions de simulations, d'où le titre de cet article, c'est un peu, un peu naïf peut être mais enfin qui marque la bifurcation. \textbf{JR} Parce que du coup oui, dans cette nouvelle dynamique, c'est assez impressionnant, du coup c'est des vrais progrès en informatique, des articles qui peuvent être publiés dans des revues d'informatique pure, sur des nouveaux algorithmes, tout le travail de, avec Mathieu\ldots \textbf{DP} Et c'est vraiment la première fois %49min49
que on touche à ce milieu là, je pense que Stéphane Cura avait du faire une petite publication quelque part, Benoît Glisse avait aussi du tirer un article en informatique, mais à partir de sa thèse, auquel nous n'avons pas participé - \textbf{JR} D'accord oui - \textbf{DP} on a l'a fait cosigner une présentation à Oxford je crois, un congrès de Systèmes Complexes, alors que là y'a des articles qui ont été co-écrits, jusqu'au livre.

\paragraph{JR}

Oui oui oui, alors là oui du coup on est dans le présent. Alors juste, OpenMole, j'aurai un entretien avec Romain donc on va peut être pas rentrer dans le détail technique d'OpenMole - \textbf{DP} oui il parlera mieux d'OpenMole - \textbf{JR} mais quel a été le germe d'OpenMole, et est arrivé à quel moment exactement, quelles ont été les premières\ldots 

\paragraph{DP}

Alors la première fois que j'ai mentionné, moi sous ma plume, Open Scalable Platform, c'est dans le projet GeoDivercity, que j'ai donc soumis en, j'ai du le soumettre en 2009, puisque je l'ai obtenu en 2010. Et c'était sur la suggestion de Thomas Louail, qui avait déjà le vocabulaire. J'avais fait donc un, j'avais déjà inscrit comme un des projets, puisqu'on, j'étais déjà dans les accointances de l'ISC, je sais plus, j'ai pas noté ces affaires là, mais ça faisait déjà des années que j'avais croisé Paul Bourgine, travailler dans les conditions qui devaient aboutir à la création de l'ISC, et j'étais peut être déjà au comité directeur de l'ISC, donc je voyais bien qu'il fallait aller dans ce sens là, mais je dois dire que la plateforme n'était pas vraiment, je sais pas sous quelle\ldots Est-ce que Thomas m'en a parlé parce qu'il y avait déjà, ils avaient déjà fait leur exercice camembert là, Romain et Mathieu, je sais pas - \textbf{JR} Ils avaient déjà commencé quelque chose avant - \textbf{DP} Ils ont fait quelque chose sur le camembert d'Isigny (rires) c'est lui qui en parle comme ça quand ils en font l'histoire, il faut demander à Romain, c'est une histoire assez amusante. En tout cas ça été vraiment la contingence, et la rencontre de trajectoires qui, au bon moment, y'avait le financement, y'avait l'intérêt, et on est partis, et avec donc trois doctorants géographes capables d'embrayer dessus direct - \textbf{JR} Clara, Clémentine \ldots - \textbf{DP} Clara, Clémentine et Sébastien, Sébastien participant à SimProcess, en tant que géomaticien allant dans les programmes, sur les transferts de données entre les bases de données géographiques, pouvant discuter les sorties de visualisation, des choses comme ça. Oh que Hélène Mathian savait faire un peu avant, mais à un degré moindre, et en tout cas pas au jour le jour, permettant\ldots 

\paragraph{JR}

Oui, alors ça c'est génial de pouvoir arriver à un, cette indépendance du thématicien, dans l'exploration du modèle et même dans l'écriture du modèle, alors du coup j'espère que pour le futur on aura\ldots Alors moi j'essaye d'enseigner ça à mes élèves, dès le début, qu'ils peuvent, même si c'est pas des programmeurs, déjà avoir leur indépendance, c'est assez remarquable qu'il puisse y avoir comme ça des outils\ldots


\paragraph{DP}

Oui c'est ce que fait Arnaud aussi, enfin Arnaud avec les formations qu'il a faites, en incitant les gens à se mettre ensemble et faire un petit programme NetLogo, pour montrer qu'ils en étaient capables, c'est exactement ce qu'il faudrait faire. C'est vraiment, enfin moi je trouve que c'est donner aux jeunes, essayer de donner aux jeunes l'autonomie que nous on donnait en les emmenant au centre de calcul, avec des cartes perforées toutes prêtes, pour passer des analyses multivariées, c'est vraiment mettre à disposition l'instrument nouveau [?] appuyant le, appuyant la recherche par le, comment dire, la méthode en pointe pour dire quelque chose à partir des chiffres, des régularités, des observations, transformer les données en résumé, résultat significatif. - \textbf{JR} Donc oui les outils et les méthodes de pointe\ldots \textbf{DP} Oui

\paragraph{JR}

\ldots au service de\ldots d'accord. Donc là on a à peut près fait le présent. Et du coup, plus sur le futur, alors est-ce que déjà on peut dire que le livre de Geodivercity c'est une sorte d'aboutissement de la théorie évolutive, c'est est-ce que c'est, déjà c'est très mature

\paragraph{DP}

C'est un jalon très très important parce que il montre la pertinence et faisabilité. - \textbf{JR} Oui. - \textbf{DP} Et la chance a été aussi de pouvoir réunir dans le programme non seulement donc cet arsenal méthodologique de la plateforme OpenMole, et de compétences intellectuelles qui vont avec, mais aussi des gens prêts à se coltiner le travail de construction de bases de données, sur des pays très vastes et différents, Brésil, Chine, Inde, Europe, Etats-Unis, bon Afrique du Sud c'est tout petit heureusement, mais très coton à construire en bases de données, dans la mesure où après l'Apartheid, toutes les circonscriptions administratives ont été volontairement redessinées pour essayer de gommer le problème territorial, et donc y'a des problèmes d'appareillement, mais ça c'est un problème, enfin ces problèmes d'appareillement de bases de données, c'est quelque chose qu'on connaissait aussi en France, parce que pour tracer les profils socio-économiques des villes, depuis les années 50, vous avez eu trois changements de nomenclature des activités économiques, d'autres changements e nomenclature des catégories socio-professionelles, sept ou huit changements de délimitation des agglomérations ou des aires urbaines, donc dans le laboratoire, on a ténu à jour toute la liste actualisée des bases de données, en se confrontant à ces exercices qui sont vraiment pas gratifiants\ldots \textbf{JR} C'est laborieux mais c'est aussi méthodo, sur la méthode des données c'est assez pointu quand même. - \textbf{DP} Alors c'est un challenge permanent pour - \textbf{JR} Oui une vraie problématique de recherche - \textbf{DP} Oui oui, et qui peut éventuellement se perfectionner, Anne Bretagnolle avec Harmonie-cités a beaucoup perfectionné le processus\ldots \textbf{JR} Ouais donc l'aspect données à part entière - \textbf{DP} Oui oui, la construction des données c'est\ldots

\paragraph{JR}

Et donc oui, est-ce que vous auriez une idée de l'avenir, des développements futurs - \textbf{DP} Ah je suis à la retraite ! (rires) - \textbf{JR} à moyen - non mais pas personnels - moyen et long terme\ldots

\paragraph{DP}

Alors le futur c'est, alors je voudrais déjà qu'il sorte quelque chose de, j'ai pas encore inventé quoi, du colloque d'octobre, qui va faire le point des travaux de Geodivercity, confrontés à l'adversité théorique - \textbf{JR} Avec des économistes qui viennent, des physiciens - \textbf{DP} Avec Michael Storper, avec Barthélémy - \textbf{JR} Avec Marc Barthélémy ! - \textbf{DP} voilà là y'a deux gros - \textbf{JR} ah, confronter les théories - \textbf{DP} voilà, des argumentaires à faire valoir pour essayer de dépasser les affrontements, de faire travailler ensemble - \textbf{JR} Oui, de\ldots - \textbf{DP} Donc ça c'est un premier point, et j'aimerais peut être mettre en place une forme de publication innovante, je sais pas si vous avez des idées\ldots \textbf{JR} mmh, alors j'aurais dit, hors du cliché de un truc interactif en ligne - \textbf{DP} y'a interactif en ligne, y'a la controverse, voyez \ldots \textbf{JR} innovante\ldots \textbf{DP} Maintenant on pourrait aussi\ldots \textbf{JR} moi je ferais bien un truc où le lecteur choisit son angle d'entrée, et ça lui agence les différents points de vue, selon, ou alors il a un graphe de lecture et il peut naviguer dedans, et selon là, on dit bah moi je veux l'entrée théorique du physicien, et bon bref le point de vue, et donc tous les points de vue sont toujours là mais selon la définition - \textbf{DP} Donc y'aurait une grappe d'article ou de vidéos, sur le blog de Geodivercity par exemple, à plusieurs entrées\ldots \textbf{JR} Ouais. Mais du coup plus de manière dynamique, je sais pas exactement comment - \textbf{DP} Faudrait des vidéos de 5min où à chaque défendeur d'une position extrême, la présente. \textbf{JR} Ouais, mais après ce qui serait pas évident c'est que ça puisse s'articuler de différentes façons, enfin peut être que dans ce cas des textes ça serait mieux\ldots \textbf{DP} Passer par des textes. Je suis en train de\ldots Il faut cette semaine que j'explique les règles du jeu aux gens, et effectivement. Mais on demande des textes courts, deux pages quoi. Je vais essayer de faire ça, si les gens sont de bonne volonté ça peut marcher. \textbf{JR} Donc ça c'est du moyen terme\ldots \textbf{DP} Ca c'est vraiment du court terme, et alors à moyen terme, bah écoutez, c'est\ldots Alors y'a ce que vous faites sur la Chine, je viens d'apprendre, on a soumis, on a répondu mais pas en coordinateurs, en partenaires - \textbf{JR} Pour Mosaic ? - \textbf{DP} A Mosaic. Je sais pas à quel moment on aura le résultat de Mosaic, mais je viens d'apprendre que, vous gardez ça pour vous, Arnaud est au courant\ldots \textbf{JR} Euh c'est enregistré\ldots \textbf{DP} Ah oui d'accord - \textbf{JR} Euh on peut peut être en parler après du coup. \textbf{DP} On en parlera après, oui oui. Oui parce que tous ces projets de recherche concrets, ça ne regarde pas l'histoire de la théorie évolutive en tant que telle. - \textbf{JR} Ou ce sera l'histoire dans 10 ans. - \textbf{DP} Oui enfin en tout cas disons que y'a des moyens de la mettre en oeuvre à différentes échelles et dans des programmes de recherche complémentaires.

\paragraph{JR}

Alors là y'a des, y'a pas mal d'idées qui germent au labo, avec Renaud LeGoix, Julien tout ça, peut être de poser une ANR ou une ERC sur une comparaison des marchés immobiliers multi-pays, mais du coup y'aurait une très grande implication du quali, parce que sur le marché immobilier il faut vraiment aller voir sur place les gens et tout ça, et en plus du quanti parce que là y'a du travail un peu à la Fabien Pfaender de collection des données, des données hétérogènes tout ça, donc là y'a un projet génial à faire, alors du quanti-quali, sur l'intégré tout ça, et alors ça je suis sûr qu'on peut lier à la théorie évolutive, parce que derrière c'est des mécaniques\ldots

\paragraph{DP}

Complètement, et la variable prix, enfin j'ai vraiment vitupéré les économistes qui ne faisaient pas leur travail d'observation empirique, de cette variable là qui représente malgré tout, c'est la valorisation des lieux, donc c'est l'importance sociale attachée au lieu, qui avec différents biais, mais quand même est matérialisée dans ce prix, et peu importe les variations et fluctuations de la conjoncture, il y a des différentiels de pris entre les villes et à l'intérieur des villes, qui sont faramineux, qui sont super intéressants à observer, et les économistes ne [?] le font\ldots \textbf{JR} Ils sont pas mal en retard les économistes non sur les big data tout ça\ldots \textbf{DP} Ah c'est assez monstrueux, mais parce que tout leur système de validation intellectuelle est orienté vers une théorisation perverse par le modèle conjectural, théorique au sens mathématique du terme. Et ça, c'est une perte d'énergie humaine et\ldots \textbf{JR} Parce que c'est extrêmement difficile\ldots \textbf{DP} Bien sûr, et puis voilà, alors qu'on pourrait faire ensemble tellement de bonnes choses, en utilisant leur savoir faire qui est étonnant, sur tout ce qui est bien sûr marché, comptable\ldots bref. En tout cas oui, travailler sur les prix c'est une très bonne idée, en plus à l'international, c'est là qu'il faut être maintenant, parce que la population, ses variations sont complètement contraintes par le stade de la transition démographique, nous sommes en train d'atteindre la transition démographique, la transition urbaine, donc des comparaisons internationales de croissances de villes, de ce point de vue ça ne veut absolument plus rien dire. En revanche le capital investit dans les villes, au sens quasi-boursier du terme, au sens des investissements industriels, au sens de la valeur reconnue aux bâtiments via les prix, c'est fondamental, et ça fait sens à l'international, et parce que ça engendre, ça conditionne des dynamiques internationales qui sont, qui prennent une importance croissante, en dépit des nouveaux colbertistes, des neo-colbertismes à la Trump. En tout cas oui oui certainement c'est dans ces directions là qu'il faut faire ces programmes de recherche maintenant.

\paragraph{JR}

En plus à l'interface des disciplines, des approches, des échelles\ldots

\paragraph{DP}

Et là effectivement, bien sûr, on va pas aller relever à la main les niveaux de prix, en relevant les agences, même avec un sondage aléatoire spatial, non vraiment il faut passer sur internet et passer par le scrapping (rires), le moissonnage des données\ldots 

\paragraph{JR} 

D'accord. Donc je pense qu'on a un peu fait le, on a eu un bon aperçu, on a un peu fait le tour, si vous avez quelque chose à ajouter. 

\paragraph{DP}

Voilà en tout cas si vous avez des questions j'y répondrais\ldots

\paragraph{JR}

Bah là je pense qu'on a bien vu un peu l'histoire et je pense qu'on peut conclure l'entretien. Merci beaucoup !

\paragraph{DP}

Parfait, merci Juste !


























\end{document}
